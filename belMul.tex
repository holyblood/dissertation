\chapter{Higher-Order Properties of Bayesian Empirical Likelihood: Multivariate
Case}

\setcounter{assumption}{0}

\section{Introduction}
This chapter considers higher order asymptotic expansion Bayesian empirical likelihood in the multivariate case.

In the univariate case, the posterior is centered at an $M$-estimator,
which guarantees the consistency under mild regularity conditions. In constrast,
in the
multivariate case, the posterior is centered at the maximum generalized
empirical likelihood estimator. This concept is generalized from 
the work of \citet{qin1994empirical}
to exponentially tilted empirical likelihood and even the Cressie-Read
case. A similar concept has been developed in \citet{newey2004higher}
. There is slight difference between their definition
and ours. However, numerical
study shows that the performance of these two are quite similar. Furthermore,
in order for the empirical likelihood sample moments to be finite,
we need more restrictive conditions to guarantee not only the consistency
of generalized empirical likelihood estimator, but also law of
the siterative logarithm. 

The organization of the remaining sections in this chapter is as follows.
In Section 2 of this paper, we consider the basic settings of the
empirical likelihood, exponentially tilted empirical likelihood, and
finally the more general Cressie-Read divergence criterion. Section
3 contains some basic lemmas pertaining to these three formulations.
While both empirical and exponentially tilted empirical likelihood
are indeed limiting cases of the general Cressie-Read formulation,
for technical reasons as as well as for the sake of transparency,
we have presented some of these lemmas separately for these three
cases. Section 4 contains the main result, namely the asymptotic expansion
of the posterior and presents a unified derivation. Some simulation
results are presented in Section 5. Section 6 contains some concluding
remarks.


\section{Basic Settings}

Suppose $X_{1},\ldots,X_{n}$ are independent and identically distributed
random vectors satisfying $Eg\left(X_{1},\theta\right)=0$, where
$\theta\in\mathbb{R}^{p}$ and $g\left(x,\theta\right)=\left(g_{1}\left(x,\theta\right),g_{2}\left(x,\theta\right),\ldots,g_{r}\left(x,\theta\right)\right)\in\mathbb{R}^{r}$.
Following \citet{qin1994empirical}, we focus on the situation $r>p$.
When $r\le p$, the posterior will still asymptotically center around
the $M$-estimator, and all the arguments are the same as the univariate case.
However, when $r>p$ , $M$-estimators may not even exist, and we need further
generalization. In this context, \citet{owen1988empirical}, formulated
empirical likelihood as a nonparametric likelihood of the form $\prod_{i=1}^{n}w_{i}\left(\theta\right)$,
where $w_{i}$ is the probability mass assigned to $X_{i}\left(i=1,\ldots,n\right)$
satisfying the constraints 
\begin{equation}
\begin{cases}
w_{i}>0,\mathrm{for\: all\: i;}\\
\sum_{i=1}^{n}w_{i}=1;\\
\sum_{i=1}^{n}w_{i}g\left(X_{i},\theta\right)=0.
\end{cases}\label{eq:constraint-el}
\end{equation}
The target is to maximize $\prod_{i=1}^{n}w_{i}$ or equivalently
$\sum_{i=1}^{n}\log w_{i}$ with respect to $w_{1},\ldots,w_{n}$
subject to the constraints given in (\ref{eq:constraint-el}). Applying
the Lagrange multiplier method, the solution turns out to be 
\begin{equation}
\hat{w}_{i}^{\mathrm{EL}}\left(\theta\right)=\frac{1}{n\left[1+\nu^{T}g\left(X_{i},\theta\right)\right]},\; i=1,2,\ldots,n,\label{eq:sol-emp-lik}
\end{equation}
where $\nu\in\mathbb{R}^{r}$, the Lagrange multiplier satisfies 
\begin{equation}
\sum_{i=1}^{n}\frac{g\left(X_{i},\theta\right)}{1+\nu^{T}g\left(X_{i},\theta\right)}=0.\label{eq:lambda-eq}
\end{equation}


It may be noted that \citet{fang2005expected,fang2006empirical} $g\left(X_{i},\theta\right)=X_{i}-\theta$. 

Closely related to the empirical likelihood is the exponentially tilted
empirical likelihood where the objective is to maximize the Shannon
entropy $-\sum_{i=1}^{n}w_{i}\log w_{i}$ still subject to the constraints
in (\ref{eq:constraint-el}). The resulting solution is given by 
\begin{equation}
\hat{w}_{i}^{\mathrm{ET}}\left(\theta\right)=\frac{\exp\left(-\nu^{T}g\left(X_{i},\theta\right)\right)}{\sum_{j=1}^{n}\exp\left(-\nu^{T}g\left(X_{i},\theta\right)\right)},\label{eq:sol-weight-etel}
\end{equation}
where $\nu$, the Lagrange multiplier, satisfies 
\begin{equation}
\sum_{i=1}^{n}\exp\left(-\nu^{T}g\left(X_{i},\theta\right)\right)g\left(X_{i},\theta\right)=0.\label{eq:lag-mul-exp-tilt-el}
\end{equation}
The exponentially tilted empirical likelihood is related to Kullback-Leibler
divergence between two empirical distributions, one with weights $w_{i}$
assigned to the $n$ sample points, and the other with uniform weights
$1/n$ assigned to the sample points. 

The general Cressie-Read divergence criterion given by 
\[
\mathrm{CR}\left(\lambda\right)=\frac{2}{\lambda\left(\lambda+1\right)}\sum_{i=1}^{n}\left[\left(nw_{i}\right)^{-\lambda}-1\right].
\]
We focus on the cases $\lambda\ge0$ and $\lambda\le-1$, because
in these cases, $\mathrm{CR}\left(\lambda\right)$ is a convex function
of the $w_{i}\left(i=1,\ldots,n\right)$,hence the minimization
problem will produce a unique solution. The following lemma also shows
within this range, the resulting empirical weights behaviour more
like a likelihood. The limiting cases $\lambda\rightarrow0$ and $\lambda\rightarrow-1$
correspond to the usual empirical likelihood and the exponentially
tilted empirical likelihood as defined earlier. 

For convex $\mathrm{CR}\left(\lambda\right)$, its minimum will be
attained in the compact set $H_{n}$ determined by the data. The Lagrange
multiplication method now gives the weights 
\begin{equation}
\hat{w}_{i}^{\mathrm{CR}}\left(\theta\right)=\frac{1}{n}\left(\mu+\nu^{T}g\left(X_{i},\theta\right)\right)^{-\frac{1}{\lambda+1}},\, i=1,2,\ldots,n,\label{eq:weight-cr-el}
\end{equation}
where $\mu\equiv\mu\left(\theta\right)$ and $\nu\equiv\nu\left(\theta\right)$
satisfy 
\begin{equation}
\begin{cases}
\sum_{i=1}^{n}\left(\mu+\nu^{T}g\left(X_{i},\theta\right)\right)^{-\frac{1}{\lambda+1}}=n,\\
\sum_{i=1}^{n}\left(\mu+\nu^{T}g\left(X_{i},\theta\right)\right)^{-\frac{1}{\lambda+1}}g\left(X_{i},\theta \right)=0.
\end{cases}\label{eq:lag-mul-cr-el}
\end{equation}


We now introduce the posterior based on an empirical likelihood. The
basic ideal was first introduced by \citet{lazar2003bayesian} with
bunch of numerical examples. The intuition relies on close relationship
between empirical likelihood and empirical distribution. \citet{owen2010empirical}
formulated 105 4 the two concepts under the same optimization framework,
that is, they shared the same objective function, but the former 
was solved under parametric constraints, while the latter was not.
Considering this similarity, we can use the empirical likelihood as
a valid distribution parametrized by inferential target. To concrete
this intuition in a Bayesian framework, writing $\hat{w}_{i}\left(\theta\right)$
as generic notation for either $\hat{w}_{i}^{\mathrm{EL}}$, $\hat{w}_{i}^{\mathrm{ET}}$
or $\hat{w}_{i}^{\mathrm{CR}}$, $\pi$ with a prior probability density
function $\rho\left(\theta\right)$, with support in $H_{n}$, the
profile (pseudo) posterior is given by 
\begin{equation}
\pi\left(\theta\mid X_{1},X_{2},\ldots,X_{n}\right)=\frac{\prod_{i=1}^{n}\hat{w}_{i}\left(\theta\right)\rho\left(\theta\right)}{\int_{H_{n}}\prod_{i=1}^{n}\hat{w}_{i}\left(\theta\right)\rho\left(\theta\right)\diff\theta}.\label{eq:poster-el-expression}
\end{equation}


The main objective of this paper is to provide an asymptotic expansion
of $\pi\left(\theta\mid X_{1},X_{2},\ldots,X_{n}\right)$. This will
include in particular the Bernstein-von Mises theorem. Towards this
end, we develop a few necessary lemmas in the next section.


\section{Lemmas}

We first give an explanation of natural domain of $\theta$, under
empirical likelihood settings. In practice, some values of $\theta$
will result in an empty feasible set under constraints (\ref{eq:constraint-el}).
The which guarantees an non-empty feasible set, and thus a solution
of the optimization problem constitutes 120 natural domain of empirical
likelihood. One may questions whether the size of the natural domain
is large enough to contain the true value. The following lemma alleviates
this worry.
\begin{lem}
\label{lem:nondecreasing-compact-natural-domain}Assume $g\left(\cdot,\cdot\right)$
is a continuous vector value function, then the natural domain defined
by the constraints (\ref{eq:constraint-el}) is a compact set and
nondecreasing with respect to sample size $n$.\end{lem}
\begin{proof}
By the third constraint of (\ref{eq:constraint-el}), $\theta$ is
a continuous vector value function of $w_{1},w_{2},\ldots,w_{n}$,
but $w_{i}$ are defined on a simplex which is a compact set through
the first constraint of (\ref{eq:constraint-el}). We may recall that
continuous function maps compact sets to compact sets. Hence, $\theta$
is naturally defined on a compact set denoted by $H$.

If for any $j=1,2,\ldots,r$, $g_{j}\left(X_{i},\theta\right)$, $i=1,2,\ldots,n$,
are all non-positive or all non-negative, then the constraints (\ref{eq:constraint-el})
are violated and $H=\emptyset$. Hence ,
\begin{eqnarray*}
H & = & \left\{ \bigcup_{j=1}^{r}\left\{ \left[\bigcap_{i=1}^{n}\left(g_{j}\left(X_{i},\theta\right)\ge0\right)\right]\bigcup\left[\bigcap_{i=1}^{n}\left(g_{j}\left(X_{i},\theta\right)\le0\right)\right]\right\} \right\} ^{c}\\
 & = & \bigcap_{j=1}^{r}\left\{ \left[\bigcup_{i=1}^{n}\left(g_{j}\left(X_{i},\theta\right)\ge0\right)^{c}\right]\bigcap\left[\bigcup_{i=1}^{n}\left(g_{j}\left(X_{i},\theta\right)\le0\right)^{c}\right]\right\} .
\end{eqnarray*}
With $n$ increases, both $\bigcup_{i=1}^{n}\left(g_{j}\left(X_{i},\theta\right)\ge0\right)^{c}$
and $\bigcup_{i=1}^{n}\left(g_{j}\left(X_{i},\theta\right)\le0\right)^{c}$
will increase, so does their intersection $H$. 
\end{proof}
Although, intuitively we expect the empirical likelihood to behave
as the true likelihood, we need some theoretical support to show that
the former enjoys some of the basic properties of the latter. In particular,
we need to verify that $\nu$ and $\mu$ are smooth functions of $\theta$
and the (pseudo) Fisher Information based on the empirical likelihood
is positive. 

We first establish the positiveness of the Fisher information . We
consider the three cases separately to introduce more transparency
and continuity in our approach. 

Our first lemma shows that the Lagrange multipliers $\nu\left(\theta\right)$
and $\mu\left(\theta\right)$ are all smooth functions of $\theta$.
\begin{comment}
add condtions
\end{comment}

\begin{lem}
\label{lem:first-order-smooth-lagmul}For empirical likelihood, exponentially
tilted empirical likelihood and Cressie-Read with parameter$\left(\lambda\right)$,
the Lagrange multipliers $\nu\left(\theta\right)$ and $\mu\left(\theta\right)$
are smooth functions of $\theta.$\end{lem}
\begin{proof}
We first consider empirical likelihood and observe that, $\nu\left(\theta\right)$
is a implicit function of $\theta$ in view of (\ref{eq:lambda-eq})
. Further 
\[
\frac{\partial}{\partial\nu}\sum_{i=1}^{n}\frac{g\left(X_{i},\theta\right)}{1+\nu^{T}g\left(X_{i},\theta\right)}=-\sum_{i=1}^{n}\frac{g\left(X_{i},\theta\right)g^{T}\left(X_{i},\theta\right)}{\left(1+\nu^{T}g\left(X_{i},\theta\right)\right)^{2}},
\]
is negative definite, so that by the implicit function theorem, $\nu$
is differentiable in $\theta$. Moreover, differentiating both sides
of (\ref{eq:lambda-eq}) with respect to $\nu$, one gets
\begin{eqnarray*}
0 & = & \sum_{i=1}^{n}\frac{1}{1+\nu^{T}g\left(X_{i},\theta\right)}\frac{\partial g\left(X_{i},\theta\right)}{\partial\theta^{T}}\frac{\partial\theta}{\partial\nu}-\sum_{i=1}^{n}\frac{\nu^{T}g\left(X_{i},\theta\right)}{\left(1+\nu^{T}g\left(X_{i},\theta\right)\right)^{2}}\frac{\partial g\left(X_{i},\theta\right)}{\partial\theta^{T}}\frac{\partial\theta}{\partial\nu}\\
 &  & -\sum_{i=1}^{n}\frac{g\left(X_{i},\theta\right)g^{T}\left(X_{i},\theta\right)}{\left(1+\nu^{T}g\left(X_{i},\theta\right)\right)^{2}},
\end{eqnarray*}
which on simplification leads to 
\begin{equation}
\frac{\partial\theta}{\partial\nu}=-\left[\sum_{i=1}^{n}\frac{1}{\left(1+\nu^{T}g\left(X_{i},\theta\right)\right)^{2}}\frac{\partial g\left(X_{i},\theta\right)}{\partial\theta}\right]^{-1}\sum_{i=1}^{n}\frac{g\left(X_{i},\theta\right)g^{T}\left(X_{i},\theta\right)}{\left(1+\nu^{T}g\left(X_{i},\theta\right)\right)^{2}},\label{eq:decreasing-theta-to-nu-2}
\end{equation}


Next, for exponentially tilted empirical likelihood, in view of (\ref{eq:lag-mul-exp-tilt-el})
and the relation 
\begin{eqnarray*}
 &  & \frac{\partial}{\partial\nu}\left(\sum_{i=1}^{n}\exp\left(-\nu^{T}g\left(X_{i},\theta\right)\right)g\left(X_{i},\theta\right)\right)\\
 & = & -\sum_{i=1}^{n}\exp\left(-\nu^{T}g\left(X_{i},\theta\right)\right)g\left(X_{i},\theta\right)g^{T}\left(X_{i},\theta\right).
\end{eqnarray*}
Note that this matrix is negative definite. Once again, the implicit
function theorem guarantees the differentiability of $\nu$ in $\theta$.
Further, differentiating both sides of (\ref{eq:lag-mul-exp-tilt-el})
with respect to $\theta$, one gets 
\begin{equation}
\frac{\partial\nu}{\partial\theta}=\left(\sum_{i=1}^{n}\frac{g\left(X_{i},\theta\right)g^{T}\left(X_{i},\theta\right)}{\exp\left(-\nu^{T}g\left(X_{i},\theta\right)\right)}\right)^{-1}\left(\sum_{i=1}^{n}\frac{1-\nu^{T}g\left(X_{i},\theta\right)}{\exp\left(-\nu^{T}g\left(X_{i},\theta\right)\right)}\frac{\partial g\left(X_{i},\theta\right)}{\partial\theta}\right).\label{eq:first-deri-lag-mult-exp-tilted-el}
\end{equation}


A similar conclusion is achieved for $\nu\left(\theta\right)$ and
$\mu\left(\theta\right)$ defined in (\ref{eq:lag-mul-cr-el}) in
connection with CR$\left(\lambda\right)$. Specifically, defining
\[
\begin{cases}
F_{1}=\sum_{i=1}^{n}\left(\mu+\nu^{T}g\left(X_{i},\theta\right)\right)^{-\frac{1}{\lambda+1}}-n,\\
F_{2}=\sum_{i=1}^{n}\left(\mu+\nu^{T}g\left(X_{i},\theta\right)\right)^{-\frac{1}{\lambda+1}}g\left(X_{i},\theta\right),
\end{cases}
\]
it follows that,
\[
\frac{\partial\left(F_{1},F_{2}\right)}{\partial\left(\mu,\nu\right)}=-\frac{1}{\lambda+1}\left(\begin{array}{cc}
\sum_{i=1}^{n}q_{i} & \sum_{i=1}^{n}q_{i}g\left(X_{i},\theta\right)\\
\sum_{i=1}^{n}q_{i}g^{T}\left(X_{i},\theta\right) & \sum_{i=1}^{n}q_{i}g\left(X_{i},\theta\right)g^{T}\left(X_{i,},\theta\right)
\end{array}\right),
\]
 where $q_{i}=\left(\mu+\nu^{T}g\left(X_{i},\theta\right)\right)^{-\frac{1}{\lambda+1}-1}$
. Then the determinant of Jacobian is 
\begin{eqnarray*}
\det\frac{\partial\left(F_{1},F_{2}\right)}{\partial\left(\mu,\nu\right)} & = & \left(\frac{1}{\lambda+1}\right)^{2}\left(\sum_{i=1}^{n}q_{i}\sum_{i=1}^{n}q_{i}g\left(X_{i},\theta\right)g^{T}\left(X_{i},\theta\right)-\sum_{i=1}^{n}q_{i}g\left(X_{i},\theta\right)\sum_{i=1}^{n}q_{i}g^{T}\left(X_{i},\theta\right)\right)\\
 & = & \left(\frac{1}{\lambda+1}\right)^{2}\left(\sum_{i=1}^{n}q_{i}\right)^{2}\sum_{i=1}^{n}\frac{q_{i}}{\sum_{j=1}^{n}q_{j}}\left[g\left(X_{i},\theta\right)-\left(\sum_{i=1}^{n}\frac{q_{i}}{\sum_{j=1}^{n}q_{j}}g\left(X_{i},\theta\right)\right)\right]\\
 &  & \times\left[g\left(X_{i},\theta\right)-\left(\sum_{i=1}^{n}\frac{q_{i}}{\sum_{j=1}^{n}q_{j}}g\left(X_{i},\theta\right)\right)\right]^{T},
\end{eqnarray*}
which is positive definite. Again, by implicit function theorem, one
gets differentiability of $\mu\left(\theta\right)$ and $\nu\left(\theta\right)$with
respect to $\theta$, and 
\begin{eqnarray}
 &  & \left(\begin{array}{c}
\partial\mu/\partial\theta\\
\partial\nu/\partial\theta
\end{array}\right)\nonumber \\
 & = & \left(\frac{\partial\left(F_{1},F_{2}\right)}{\partial\left(\mu,\nu\right)}\right)^{-1}\left(\begin{array}{c}
\partial F_{1}/\partial\theta\\
\partial F_{2}/\partial\theta
\end{array}\right)\\
 & = & \left(-\frac{1}{\lambda+1}\right)\left(\lambda+1\right)^{2}\left[\sum_{i=1}^{n}q_{i}\sum_{i=1}^{n}q_{i}g\left(X_{i},\theta\right)g^{T}\left(X_{i},\theta\right)-\sum_{i=1}^{n}q_{i}g\left(X_{i},\theta\right)\sum_{i=1}^{n}q_{i}g^{T}\left(X_{i},\theta\right)\right]^{-1}\nonumber \\
 &  & \times\left(\begin{array}{cc}
\sum_{i=1}^{n}q_{i}g\left(X_{i},\theta\right)g^{T}\left(X_{i},\theta\right) & -\sum_{i=1}^{n}q_{i}g\left(X_{i},\theta\right)\\
-\sum_{i=1}^{n}q_{i}g^{T}\left(X_{i},\theta\right) & \sum_{i=1}^{n}q_{i}
\end{array}\right)\nonumber \\
 &  & \times\left(\begin{array}{c}
-\left(\lambda+1\right)^{-1}\sum_{i=1}^{n}q_{i}\nu^{T}\partial g\left(X_{i},\theta\right)/\partial\theta\\
-\left(\lambda+1\right)\sum_{i=1}^{n}q_{i}\nu^{T}g\left(X_{i},\theta\right)\partial g\left(X_{i},\theta\right)/\partial\theta+\sum_{i=1}^{n}\left(\mu+\nu^{T}g\left(X_{i},\theta\right)\right)^{\frac{1}{\lambda+1}}\partial g\left(X_{i},\theta\right)/\partial\theta
\end{array}\right)\label{eq:first-der-lag-mul-crel}
\end{eqnarray}

\end{proof}
The next result shows that all the derivatives of the Lagrange multipliers
$\nu\left(\theta\right)$ and $\mu\left(\theta\right)$ are smooth
functions of $\theta\in H$. We provide a unified proof for all three
cases where we utilize the previous lemma. 
\begin{assumption}
\label{assu:high-derive-cond}Assume for any $\left(k_{1},k_{2},\ldots,k_{p}\right)\in\mathbb{N}^{p}$,
satisfying $\sum_{i=1}^{p}k_{i}=k\le K+4$, the higher-order mixture
partial derivatives 
\[
\frac{\partial^{k}g\left(X,\theta\right)}{\partial\theta_{1}^{k_{1}}\partial\theta_{2}^{k_{2}}\cdots\partial\theta_{p}^{k_{p}}},
\]
exists and continuous in $\theta$, almost surely in $X$.\end{assumption}
\begin{lem}
\label{lem:mul-el-smooth-lagrange-mul-1}Under the Assumption \ref{assu:high-derive-cond},
all partial derivatives of $\nu\left(\theta\right)$ and $\mu\left(\theta\right)$
are smooth functions of $\theta$ for $\theta\in H$.\end{lem}
\begin{proof}
The result is proved by induction. We have see already in Lemma \ref{lem:first-order-smooth-lagmul},
the gradient $\nabla\nu\left(\theta\right)$ and $\nabla\mu\left(\theta\right)$
are smooth functions of $\theta$. Suppose the result holds for all
$k$th partial derivatives of $\nu\left(\theta\right)$ and $\mu\left(\theta\right)$
for $k\le K$. The writing 
\[
\nabla^{k}\nu\left(\theta\right)=h_{k}\left(\nu\left(\theta\right),\theta\right),1\le k\le K,
\]
 
\[
\nabla^{k+1}\nu\left(\theta\right)=\frac{\partial h_{k}}{\partial\nu^{T}}\nabla\nu+\frac{\partial h_{k}}{\partial\theta}
\]
which is also a smooth function of $\theta$ by the induction hypothesis
and Lemma 1. A similar proof works for $\mu\left(\theta\right)$.
\end{proof}

\section{Maximum Generalized Empirical Likelihood Estimator}

One important justification of maximum likelihood estimator is the
good performance substantiated by large sample theory, such as consistency
and asymptotic normality. It can be shown that the above theoretical results
can be seamlessly transferred onto estimators resulting from maximum
empirical weights. \citet{qin1994empirical} already pointed out the
validity to use empirical likelihood to define maximum empirical likelihood
estimator. We extend the same idea into more general Cressie-Read
family. 
\begin{defn}[Maximum Generalized Empirical Likelihood Estimator]
\label{def:gmele}Let $\hat{w}_{i}\left(\theta\right)$ be the empirical
weights on sample $X_{1},X_{2},\ldots,X_{n}$, generating from empirical
likelihood, exponentially tilted empirical likelihood or Cressie-Read
family. Then 
\[
\tilde{\theta}=\argmin_{\theta\in H}-\sum_{i=1}^{n}\log\hat{w}_{i}\left(\theta\right),
\]
called  maximum generalized empirical likelihood estimator.
\end{defn}
Lemma \ref{lem:mul-el-smooth-lagrange-mul-1} establishes the continuity
of empirical likelihood and Lemma \ref{lem:nondecreasing-compact-natural-domain}
reveals the compactness of $H$, then these results guarantee the
existence of $\tilde{\theta}$. So this concept is well-defined. One
advantage of this estimator over the traditional $M$-estimator is that
it can always produce a good quality estimator even when traditional
one has no solution. Before exploring the asymptotic properties of
this new estimator, we derive an equivalent formulation of  maximum
generalized empirical likelihood estimator, which may be more appropriate
for numerical computation and theoretical derivation. 

First, we define $\Psi$ functions under the three cases. In empirical
likelihood case, let 
\begin{eqnarray*}
\Psi_{1}\left(x\mid\theta,\nu\right) & = & \frac{g\left(x,\theta\right)}{1+\nu^{T}g\left(x,\theta\right)},\\
\Psi_{2}\left(x\mid\theta,\nu\right) & = & \frac{1}{1+\nu^{T}g\left(x,\theta\right)}\nu^{T}\frac{\partial g\left(x,\theta\right)}{\partial\theta}.
\end{eqnarray*}
Let $\Psi^{\mathrm{EL}}\left(x\mid\theta,\nu\right)=\left(\Psi_{1},\Psi_{2}\right)^{T}$.
For exponentially tilted empirical likelihood, let 
\begin{eqnarray*}
\Psi_{1}\left(x\mid\theta,\nu,\mu,\lambda_{1},\lambda_{2}\right) & = & \nu^{T}\frac{\partial g\left(x,\theta\right)}{\partial\theta}-\lambda_{1}\exp\left(-\mu-\nu^{T}g\left(x,\theta\right)\right)\nu^{T}\frac{\partial g\left(x,\theta\right)}{\partial\theta}\\
 &  & -\exp\left(-\mu-\nu^{T}g\left(x,\theta\right)\right)\nu^{T}\frac{\partial g\left(x,\theta\right)}{\partial\theta}\lambda_{2}^{T}g\left(x,\theta\right)+\exp\left(-\mu-\nu^{T}g\left(x,\theta\right)\right)\lambda_{2}^{T}\frac{\partial g\left(x,\theta\right)}{\partial\theta},\\
\Psi_{2}\left(x\mid\theta,\nu,\mu,\lambda_{1},\lambda_{2}\right) & = & g\left(x,\theta\right)-\lambda_{1}\exp\left(-\mu-\nu^{T}g\left(x,\theta\right)\right)g\left(x,\theta\right)-\exp\left(-\mu-\nu^{T}g\left(x,\theta\right)\right)g\left(x,\theta\right)\lambda_{2}^{T}g\left(x,\theta\right),\\
\Psi_{3}\left(x\mid\theta,\nu,\mu,\lambda_{1},\lambda_{2}\right) & = & 1-\lambda_{1}\exp\left(-\mu-\nu^{T}g\left(x,\theta\right)\right)-\exp\left(-\mu-\nu^{T}g\left(x,\theta\right)\right)\lambda_{2}^{T}g\left(x,\theta\right),\\
\Psi_{4}\left(x\mid\theta,\nu,\mu,\lambda_{1},\lambda_{2}\right) & = & \exp\left(-\mu-\nu^{T}g\left(x,\theta\right)\right)-\frac{e}{n},\\
\Psi_{4}\left(x\mid\theta,\nu,\mu,\lambda_{1},\lambda_{2}\right) & = & \exp\left(-\mu-\nu^{T}g\left(x,\theta\right)\right)g\left(x,\theta\right),
\end{eqnarray*}
where $\lambda_{1}$ and $\lambda_{2}$ are the new Lagrange multipliers
which we will define later. Let $\Psi^{\mathrm{ET}}\left(x\mid\theta,\nu,\mu,\lambda_{1},\lambda_{2}\right)=\left(\Psi_{1},\ldots,\Psi_{5}\right)^{T}.$
In the Cressie-Read case, let 
\begin{eqnarray*}
\Psi_{1}\left(x\mid\theta,\nu,\mu,\lambda_{1},\lambda_{2}\right) & = & \frac{1}{\lambda+1}\Bigg[\frac{\nu^{T}\partial g\left(x,\theta\right)/\partial\theta}{\mu+\nu^{T}g\left(x,\theta\right)}-\lambda_{1}\frac{\nu^{T}\partial g\left(x,\theta\right)/\partial\theta}{\left(\mu+\nu^{T}g\left(x,\theta\right)\right)^{1+1/\left(\lambda+1\right)}}\\
 &  & -\frac{\left(\nu^{T}\partial g\left(x,\theta\right)/\partial\theta\right)\left(\lambda_{2}^{T}g\left(x,\theta\right)\right)}{\left(\mu+\nu^{T}g\left(x,\theta\right)\right)^{1+1/\left(\lambda+1\right)}}+\frac{\lambda_{2}^{T}\partial g\left(x,\theta\right)/\partial\theta}{\left(\mu+\nu^{T}g\left(x,\theta\right)\right)^{1+1/\left(\lambda+1\right)}}\Bigg],\\
\Psi_{2}\left(x\mid\theta,\nu,\mu,\lambda_{1},\lambda_{2}\right) & = & \frac{1}{\lambda+1}\left[\frac{g\left(x,\theta\right)}{\mu+\nu^{T}g\left(x,\theta\right)}-\lambda_{1}\frac{g\left(x,\theta\right)}{\left(\mu+\nu^{T}g\left(x,\theta\right)\right)^{1+1/\left(\lambda+1\right)}}-\frac{g\left(x,\theta\right)\left(\lambda_{2}^{T}g\left(x,\theta\right)\right)}{\left(\mu+\nu^{T}g\left(x,\theta\right)\right)^{1+1/\left(\lambda+1\right)}}\right],\\
\Psi_{3}\left(x\mid\theta,\nu,\mu,\lambda_{1},\lambda_{2}\right) & = & \frac{1}{\lambda+1}\Bigg[\frac{1}{\mu+\nu^{T}g\left(x,\theta\right)}-\lambda_{1}\left(\mu+\nu^{T}g\left(x,\theta\right)\right)^{-1-1/\left(\lambda+1\right)}\\
 &  & -\left(\mu+\nu^{T}g\left(x,\theta\right)\right)^{-1-1/\left(\lambda+1\right)}\lambda_{2}^{T}g\left(x,\theta\right)\Bigg],\\
\Psi_{4}\left(x\mid\theta,\nu,\mu,\lambda_{1},\lambda_{2}\right) & = & \left(\mu+\nu^{T}g\left(x,\theta\right)\right)^{-1/\left(\lambda+1\right)}-1,\\
\Psi_{5}\left(x\mid\theta,\nu,\mu,\lambda_{1},\lambda_{2}\right) & = & \left(\mu+\nu^{T}g\left(x,\theta\right)\right)^{-1/\left(\lambda+1\right)}g\left(x,\theta\right).
\end{eqnarray*}
Let $\Psi^{\mathrm{CR}}\left(x\mid\theta,\nu,\mu,\lambda_{1},\lambda_{2}\right)=\left(\Psi_{1},\ldots,\Psi_{5}\right)^{T}$.
Use above definitions, we elicit the following lemma.
\begin{lem}
\label{lem:extend-m-estimator}The  maximum generalized empirical
likelihood estimator is the solution of $n^{-1}\sum_{i=1}^{n}\Psi\left(X_{i}\mid\theta,\nu,\mu,\lambda_{1},\lambda_{2}\right)=0$.
If we denote the solution as $\left(\theta^{*},\nu^{*},\mu^{*},\lambda_{1}^{*},\lambda_{2}^{*}\right)$,
then $\tilde{\theta}=\theta^{*}$, $\nu\left(\tilde{\theta}\right)=\nu^{*}$
and $\mu\left(\tilde{\theta}\right)=\mu^{*}$.\end{lem}
\begin{proof}
We provide the proofs separately for the three kinds
of empirical likelihoods. 

First for empirical likelihood, the relationship between $\theta$
and $\nu$ are restricted by equation constraint (\ref{eq:lambda-eq}).
By the Definition \ref{def:gmele}, the maximum generalized empirical
likelihood estimator can also be the solution of 
\[
\max_{\theta,\nu}l\left(\theta\right)=-\frac{1}{n}\sum_{i=1}^{n}\log\left(1+\nu^{T}g\left(X_{i},\theta\right)\right),
\]
subject to
\[
\sum_{i=1}^{n}\frac{g\left(X_{i},\theta\right)}{1+\nu^{T}g\left(X_{i},\theta\right)}=0.
\]
Directly calculation shows 
\[
\frac{\partial l\left(\theta\right)}{\partial\nu}=\frac{1}{n}\sum_{i=1}^{n}\frac{g\left(X_{i},\theta\right)}{1+\nu^{T}g\left(X_{i},\theta\right)}=\frac{1}{n}\sum_{i=1}^{n}\Psi_{1}\left(X_{i}\mid\theta,\nu\right).
\]
So the solution of unconstrained problem automatically satisfies the
constraint (\ref{eq:lambda-eq}). Moreover, 
\[
\frac{\partial l\left(\theta\right)}{\partial\theta}=\frac{1}{n}\sum_{i=1}^{n}\Psi_{2}\left(X_{i}\mid\theta,\nu\right).
\]
So the system of equations $n^{-1}\sum_{i=1}^{n}\Psi\left(X_{i}\mid\theta,\nu\right)=0$
is a necessary condition of the optimization problem.
Hence the lemma holds for the empirical likelihood case.

Next, we consider exponentially tilted empirical likelihood. Reviewing
the Lagrange method to get the empirical weights (\ref{eq:sol-weight-etel}),
we find 
\[
\hat{w}_{i}\left(\theta\right)=\exp\left(-1-\mu-\nu^{T}g\left(X_{i},\theta\right)\right),
\]
where $\mu$ and $\nu$ satisfy 
\begin{eqnarray*}
\sum_{i=1}^{n}\exp\left(-1-\mu-\nu^{T}g\left(X_{i},\theta\right)\right) & = & 1,\\
\sum_{i=1}^{n}\exp\left(-1-\mu-\nu^{T}g\left(X_{i},\theta\right)\right)g\left(X_{i},\theta\right) & = & 0.
\end{eqnarray*}
 Hence, the optimization problem in Definition \ref{def:gmele} shares
the same maximum as the problem 
\[
\max_{\theta,\nu,\mu}l\left(\theta\right)=\frac{1}{n}\sum_{i=1}^{n}\left(-1-\mu-\nu^{T}g\left(X_{i},\theta\right)\right),
\]
subject to 
\begin{eqnarray*}
\sum_{i=1}^{n}\exp\left(-1-\mu-\nu^{T}g\left(X_{i},\theta\right)\right) & = & 1,\\
\sum_{i=1}^{n}\exp\left(-1-\mu-\nu^{T}g\left(X_{i},\theta\right)\right)g\left(X_{i},\theta\right) & = & 0.
\end{eqnarray*}
Unlike the empirical likelihood case, here we do not have any shortcut.
The Lagrange multiplier method is the only choice. The Lagrangian
can be written as 
\begin{eqnarray*}
L\left(\theta,\nu,\mu,\lambda_{1},\lambda_{2}\right) & = & \frac{1}{n}\sum_{i=1}^{n}\left(-1-\mu-\nu^{T}g\left(X_{i},\theta\right)\right)+\lambda_{1}\left(\sum_{i=1}^{n}\exp\left(-\mu-\nu^{T}g\left(X_{i},\theta\right)\right)-e\right)\\
 &  & +\lambda_{2}^{T}\left(\sum_{i=1}^{n}\exp\left(-1-\mu-\nu^{T}g\left(X_{i},\theta\right)\right)g\left(X_{i},\theta\right)\right).
\end{eqnarray*}
Calculation shows $\nabla L=n^{-1}\sum_{i=1}^{n}\Psi\left(X_{i}\mid\theta,\nu,\mu,\lambda_{1},\lambda_{2}\right)$.
Hence, the same argument supporting empirical likelihood case validate
the lemma in exponentially tilted empirical likelihood case.

The Cressie-Read case can follow exactly the same procedure in exponentially
tilted empirical likelihood case.
\end{proof}
This lemma offer a better numerical scheme to get the maximum generalized
empirical likelihood estimator than using the original definition. In the original
definition, for each iteration of $\theta$, one needs to solve nonlinear
equations to get Lagrange multipliers $\nu$, which will introduce
another iteration. By using the first order condition as in this lemma,
we can use usual nonlinear equation solver to get the result in a
single layer iteration.  Lemma \ref{lem:extend-m-estimator}
also provides a quick solution on asymptotic properties of maximum
generalized empirical likelihood estimator. Indeed, it is trivial
to see, 
\begin{eqnarray*}
E\Psi^{\mathrm{EL}}\left(X\mid\theta_{0},0\right) & = & 0,\\
E\Psi^{\mathrm{ET}}\left(X\mid\theta_{0},0,-1,0\right) & = & 0,\\
E\Psi^{\mathrm{CR}}\left(X\mid\theta_{0},0,1,0,1\right) & = & 0.
\end{eqnarray*}
Hence, maximum generalized empirical likelihood estimator can also
be interpreted as an ordinary $M$-estimator defined by $\Psi$ function.
The consistency and asymptotic normality can  easily be extracted from
the ordinary $M$-estimator theory. However, in order to expand the posterior
around maximum generalized empirical likelihood estimator, we need
a slight stronger asymptotic property called the law of the iterative logarithm.
Particularly, the consistency requires the following condition from 
\citet{huber1967behavior}.
\begin{assumption}
\label{assu:consistency-m-est}%
Let $\eta=\left(\theta,\nu,\mu,\lambda_{1},\lambda_{2}\right)$, $\psi\left(\eta\right)=E\Psi\left(X\mid\eta\right)$
	\begin{enumerate}
	\item Let the parameter space $H$ be locally compact with countable base.
	\item For each fixed $\eta\in H$, $\Psi\left(x\mid\eta\right)$ is square
	integrable in $\eta$ and separable in sense of Doob: there is zero
	measure set $N$ and a countable subset of $H'\subset H$, such that
	for every open set $U\subset H$, and every closed interval $A$,
	the sets $\left\{ x:\Psi\left(x\mid\eta\right)\in A,\:\forall\eta\in U\right\} $
	and $\left\{ x:\Psi\left(x\mid\eta\right)\in A,\:\forall\eta\in U\cap H'\right\} $
	differ by a subset of $N$.
	\item The function $\Psi$ is a.s. continuous in $\theta$. 
	\item $\psi$ has unique zero at $\eta=\eta_0$.
	\item There exists a continuous function which is bounded away from zero, 
			$b(\eta)>b_0>0$, such that
			\begin{enumerate}
				\item $\sup_\eta b^{-1}(\eta)\left|\Psi(x,\eta)\right|$ is 	 
					integrable,
				\item $\varliminf_{\eta \rightarrow \infty} b^{-1}(\eta) \left|
					\psi(\eta)\right| \ge 1$,
				\item $E\left\{ \varlimsup_{\eta \rightarrow \infty} b^{-1}(\eta)
						\left|\Psi(\eta)-\psi(\eta)\right|\right\} <1$.
			\end{enumerate}
	\end{enumerate}
\end{assumption}
The assumptions we need come from \citet{he1995law}. We state them
as follows.
\begin{assumption}
\label{assu:lil-m-est}
Let $u\left(x,\eta,d\right)=\sup_{\left|\tau-\eta\right|\le d}\left|\Psi\left(x\mid\tau\right)-\Psi\left(x\mid\eta\right)\right|$,
where $\left|\cdot\right|$ takes sup-norm $\left|\eta\right|=\max_{1\le j\le p}\left|\eta_{j}\right|$.
Let $\eta_{n}$ satisfy 
\[
\frac{1}{\sqrt{n\log\log n}}\sum_{i=1}^{n}\Psi\left(X_{i}\mid\eta_{n}\right)\rightarrow0,\ascv.
\]
\end{assumption}
The following conditions guarantee law of the iterative logarithm.
\begin{assumption}
\label{assu:lil-m-est-2}
\begin{enumerate}

\item There is a $\eta_{0}\in H$, such that $\psi\left(\eta_{0}\right)=0$,
and $\psi$ has a non-singular derivative at $\eta_{0}$.
\item There exist positive numbers $a,\: b,\: c,\: d,\:\alpha,\:\beta$,
and $d_{0}$ such that $\alpha\ge\beta>2$, and 

\begin{enumerate}
\item $\left|\psi\left(\eta\right)\right|\ge a\left|\eta-\eta_{0}\right|$,
for $\left|\eta-\eta_{0}\right|\le d_{0}$,
\item $Eu\left(x,\eta,d\right)\le bd$, for $\left|\eta-\eta_{0}\right|+d\le d_{0}$,
\item $Eu^{\alpha}\left(x,\eta,d\right)\le cd^{\beta}$, for $\left|\eta-\eta_{0}\right|+d\le d_{0}$.
\end{enumerate}
\item $\left|\eta_{n}-\eta_{0}\right|\le d_{0}$ almost surely as $n$ goes
infinity.

\end{enumerate}
\end{assumption}
Actually, the second condition is merely requiring we have a theoretical
target, and the fourth condition is automatically satisfied if we
have consistency. By the theorem in \citet{he1995law},
we get the law of the iterative logarithm for both maximum generalized
empirical likelihood estimator and the Lagrange multipliers. Now we
are ready to state another lemma to tie the empirical likelihood weights
and likelihood together with following condition on the unknown distribution.
\begin{assumption}
\label{assu:finite-theoretic-moment} Let $\left(k_{j1},k_{j2},\ldots,k_{jp}\right)\in\mathbb{N}^{p}$,
and $\sum_{i=1}^{p}k_{ji}=k_{j}\le K+4$, $j=1,\ldots,l$, then 
\[
E\left(\prod_{j=1}^{l}\frac{\partial^{k_{l}}g\left(X,\theta_{0}\right)}{\partial\theta_{1}^{k_{l1}}\partial\theta_{2}^{k_{l2}}\cdots\partial\theta_{p}^{k_{lp}}}\right),
\]
exists and is finite.\end{assumption}
\begin{lem}
\label{lem:finite-empirical-weight-moment}Let $\omega_{i}\left(\theta\right)$
be the unnormalized empirical weights from empirical likelihood, that
is $\omega_{i}\left(\theta\right)=n\hat{w}_{i}\left(\theta\right)$
in empirical likelihood or Cressie-Read case, and $\omega_{i}\left(\theta\right)=\exp\left(-\nu\left(\theta\right)g\left(X_{i},\theta\right)\right)$
in exponentially tilted empirical likelihood. Use the same notation
in Assumption \ref{assu:high-derive-cond}. Under the Assumption \ref{assu:high-derive-cond},
\ref{assu:consistency-m-est}, \ref{assu:lil-m-est}, \ref{assu:lil-m-est-2} and \ref{assu:finite-theoretic-moment},
then for any $k\le K+4$, any number $s\in\mathbb{R}$, 
\[
\lim_{n\rightarrow\infty}\frac{1}{n}\sum_{i=1}^{n}\omega_{i}\left(\tilde{\theta}\right)^{s}\prod_{j=1}^{l}\frac{\partial^{k_{l}}g\left(X,\theta_{0}\right)}{\partial\theta_{1}^{k_{l1}}\partial\theta_{2}^{k_{l2}}\cdots\partial\theta_{p}^{k_{lp}}}=E\left(\prod_{j=1}^{l}\frac{\partial^{k_{l}}g\left(X,\theta_{0}\right)}{\partial\theta_{1}^{k_{l1}}\partial\theta_{2}^{k_{l2}}\cdots\partial\theta_{p}^{k_{lp}}}\right)\ascv.
\]
\end{lem}
\begin{proof}
Similar to%
\begin{comment}
add detail lemma number
\end{comment}
{} \citet{owen2010empirical}, we have 
\[
\lim_{n\rightarrow\infty}\max_{1\le i\le n}n^{-1/3}g\left(X_{i},\theta_{0}\right)=0\ascv.
\]
By law of the iterative logarithm on Lagrange multiplier $\nu$, there
exists some constant $C_{1}$ such that 
\[
\varlimsup_{n\rightarrow\infty}\frac{\sqrt{n}\nu\left(\tilde{\theta}\right)}{\sqrt{2\log\log n}}=C_{1}\ascv.
\]
Hence 
\[
\varlimsup_{n\rightarrow\infty}\nu^{T}\left(\tilde{\theta}\right)\max_{1\le i\le n}g\left(X_{i},\tilde{\theta}\right)=o\left(\sqrt{\frac{2\log\log n}{n}}\times n^{1/3}\right)=o\left(\frac{\sqrt{2\log\log n}}{n^{1/6}}\right)=0.
\]
So $\nu^{T}\left(\tilde{\theta}\right)g\left(X_{i},\tilde{\theta}\right)$
are uniformly going to zero. Hence $\omega_{i}\left(\tilde{\theta}\right)$
are uniformly going to 1 and the lemma holds.
\end{proof}
This lemma supports consistency of empirical sample moment to the true
moment and further justifies the intuition to use empirical likelihood
as a valid likelihood. 

\citet{newey2004higher} proposed a similar estimator replacing the
objective function in Definition \ref{def:gmele} with divergence
measure when one calculates empirical weights. Their definition coincides
with us when the divergence measure is empirical likelihood. Although,
their starting point is generalized method of moment estimator and
the intuition is slight different from ours, the simulation hardly differs
the performance. In Table \ref{tab:mgele-mde}, we present the simulation
result. 
\begin{table}
% latex table generated in R 3.1.0 by xtable 1.7-3 package % Fri Apr 03 14:30:16 2015
%\begin{table}[ht]
\centering
\begin{tabular}{rrrrr}
  \hline
 & MGELE ETEL & MDE ETEL & MGELE CR & MDE CR \\ 
  \hline
N=20 & 0.1394 & 0.1320 & 0.2130 & 0.1408 \\
   N=50 & 0.0759 & 0.0759 & 0.1867 & 0.0766 \\
   N=100 & 0.0536 & 0.0536 & 0.2099 & 0.0539 \\
   N=200 & 0.0392 & 0.0401 & 0.2214 & 0.0396 \\
   N=500 & 0.0274 & 0.0298 & 0.1824 & 0.0274 \\
    \hline
\end{tabular}
%\end{table}\protect\caption{\label{tab:mgele-mde}Compare Performance MGELE and MDE}
\caption{\label{tab:mgele-mde}Compare Performance MGELE and MDE}
\end{table}


Before stating the main result, 
we need to prove that the Hessian matrices of the empirical likelihood, the exponentially tilted empirical likelihood and Cressie-Read empirical likelihood are positive definite.

\begin{lem}
\label{lem:pd-sample-var-2}
Under  Assumptions \ref{ass:denom-not-zero} and \ref{ass:first-order-smooth-g}, 
there exists an $N$, 
such that for any $n\ge N$,
${\partial^{2}\tilde{l}_n\left(\tilde{\theta} \right)} / {\partial\theta \partial\theta^T}$ is negative definite,
where $\tilde{l}\left(\theta\right)=n^{-1}\sum_{i=1}^{n}\log\hat{w}_{i}\left(\theta\right)$
where $\hat{w}_{i}$ is either $\hat{w}_{i}^{\mathrm{EL}}$, $\hat{w}_{i}^{\mathrm{ET}}$
or $\hat{w}_{i}^{\mathrm{CR}}$ $\left(i=1,\ldots,n\right)$.
\end{lem}

\begin{proof}
	See Appendix B.1 
\end{proof}



\section{Main Result\label{sec:main-result}}

For multivariate case
in the empirical likelihood, exponentially tilted empirical likelihood and Cressie-Read empirical likelihood, let $B$ be the Cholesky decomposition of
negative of the Hessian matrix 
\[
-\left.\frac{\partial^{2}\hat{l}}{\partial\theta^{T}\partial\theta}\right|_{\theta=\tilde{\theta}}=B^{T}B.
\]
Let $Y=\sqrt{n}B\left(\theta-\tilde{\theta}\right)$, define differential
operators 
\[
\delta_{i}=\frac{1}{i!}\left(Y^{T}B^{-T}\nabla\right)^{i},
\]
where $\nabla$ is the gradient operator. 

We expand the prior around the maximum generalized empirical likelihood estimator $\tilde{\theta}$ and have
\begin{eqnarray*}
\rho_{K}\left(\theta\right) & = & \rho\left(\tilde{\theta}\right)+\delta_{1}\rho\left(\tilde{\theta}\right)n^{-\frac{1}{2}}+\delta_{2}\rho\left(\tilde{\theta}\right)n^{-1}+\cdots+\delta_{K}\rho\left(\tilde{\theta}\right)n^{-\frac{K}{2}},
\end{eqnarray*}
where $\delta_{i}\rho\left(\tilde{\theta}\right)$ are results from
applying the differential operators prior density then evaluating
at $\tilde{\theta}$. We abbreviate $\delta_{i}\rho\left(\tilde{\theta}\right),\: i=2,3,\ldots K$
as $\delta_{i}\rho$.

\begin{comment}
add multivariate case coefficients
\end{comment}



Let $I_{i,h}=\left\{ \left(m_{3,i},m_{4,i},\ldots,m_{K+3,i}\right)\in\mathbb{N}^{K}:\sum_{u=3}^{K+3}m_{u,i}=i,\:\sum_{u=3}^{K+3}m_{u,i}\left(u-2\right)=h\right\} $ be index sets.

\begin{comment}
the formula is wrong, need a new one. 
\end{comment}
Let $P_{K}\left(A,n\right)$ be the expansion polynomial 
\begin{eqnarray*}
 &  & P_{K}\left(A,n\right)\\
 & = & \left(\rho\left(\tilde{\theta}\right)\int_{A\cap H_{n}}\exp\left(-\frac{Y^{T}Y}{2}\right)\diff Y\right)n^{-\frac{1}{2}}+\left(\int_{A\cap H_{n}}\exp\left(-\frac{Y^{T}Y}{2}\right)\left(\delta_{1}\rho+\rho\left(\tilde{\theta}\right)\delta_{3}\hat{l}\right)\diff Y\right)n^{-1}\\
 &  & +\sum_{h=2}^{K}\Bigg\{\int_{A\cap H_{n}}\exp\left(-\frac{Y^{T}Y}{2}\right)\Bigg[\delta_{h}\rho\\
 &  & +\sum_{j=0}^{h-1}\delta_{j}\rho\sum_{\frac{h-j}{K+1}\le i\le h-j}\frac{1}{i!}\sum_{I_{i,h-j}}\binom{i}{m_{3,i},m_{4,i},\ldots,m_{K+3,i}}\prod_{u=3}^{K+3}\left(\delta_{u}\hat{l}\right)^{m_{u,i}}\Bigg]\diff Y\Bigg\} n^{-\frac{h+1}{2}}\\
 & = & \int_{A\cap H_{n}}\exp\left(-\frac{Y^{T}Y}{2}\right)\sum_{h=0}^{K}\alpha_{h}\left(Y,n\right)n^{-\frac{h}{2}}\diff n^{-\frac{1}{2}}Y,
\end{eqnarray*}
where $\alpha_{h}\left(Y,n\right)$ are corresponding coefficients
before each order of $n$. % 
The main theorem of this section is as follows.
\begin{comment}
add normal case and multivariate case
\end{comment}
For rigorous expansion, we need the $M$-estimator to be unique.
\begin{assumption}
\label{assu:uniq-m-est-mult}
Assume $\tilde{\theta}$ is the unique maximizer in $H$.
\end{assumption}
\begin{thm}[Fundamental Theorem for Expansion]
\label{thm:main-theorem-2}Under the Assumptions \ref{assu:consistency-m-est}, \ref{assu:lil-m-est}, \ref{assu:lil-m-est-2},  
\ref{assu:finite-theoretic-moment} and \ref{assu:uniq-m-est-mult},
there exist a positive constant $M_{1}$ and a large integer $N_{1}$,
such that for any Borel set $A\subset\mathbb{R}^{p}$, and any $n>N_{1}$%
\begin{comment}
add subscript to constant
\end{comment}
{} 
\[
\left|\int_{A\cap H_{n}}\prod_{i=1}^{n}\hat{w}_{i}\left(\theta\right)\pi\left(\theta\right)\diff n^{-\frac{1}{2}}Y-P_{K}\left(A,n\right)\right|\le M_{1}n^{-\frac{K+2}{2}},\ascv.
\]

\end{thm}

This theorem can be used to prove many asymptotic results
\begin{thm}[Asymptotic expansion]
Under the same assumptions as Theorem \ref{thm:main-theorem-2}, there exist
a positive constant $M_{2}$ and a large integer $N_{2}$, such that
for any Borel set $A\subset\mathbb{R}^{p}$ and any $n>N_{2}$,%
\begin{comment}
add subscript to constant
\end{comment}
{} 
\begin{equation}
\left|\Pi\left(B\left(\theta-\tilde{\theta}\right)\in A|X_{1},X_{2},\ldots,X_{n}\right)-\Phi_{p}\left(A|H_{n}\right)-\sum_{i=1}^{K}\gamma_{i}\left(A,n\right)n^{-\frac{i}{2}}\right|\le M_{1}n^{-\frac{K+1}{2}},\ascv.\label{eq:asym-exp-poster-prob}
\end{equation}

\begin{proof}
By Theorem \ref{thm:main-theorem-2}, we have 
\begin{equation}
\left|\int_{A\cap H_{n}}\prod_{i=1}^{n}\hat{w}_{i}\left(\theta\right)\pi\left(\theta\right)\diff n^{-\frac{1}{2}}Y-P_{K}\left(A,n\right)\right|\le M_{1}n^{-\frac{K+2}{2}},\label{eq:main-theorem-result-A}
\end{equation}
and 
\begin{equation}
\left|\int_{H_{n}}\prod_{i=1}^{n}\hat{w}_{i}\left(\theta\right)\pi\left(\theta\right)\diff n^{-\frac{1}{2}}Y-P_{K}\left(H_{n},n\right)\right|\le M_{1}n^{-\frac{K+2}{2}}.\label{eq:main-theorem-result-Hn}
\end{equation}
By definition 
\[
\Pi\left(B\left(\theta-\tilde{\theta}\right)\in A|X_{1},X_{2},\ldots,X_{n}\right)=\frac{\int_{A\cap H_{n}}\prod_{i=1}^{n}\hat{w}_{i}\left(\theta\right)\pi\left(\theta\right)\diff n^{-\frac{1}{2}}Y}{\int_{H_{n}}\prod_{i=1}^{n}\hat{w}_{i}\left(\theta\right)\pi\left(\theta\right)\diff n^{-\frac{1}{2}}Y}.
\]
\begin{comment}
add more detail in bounded. bounded above and below.
\end{comment}
We know that all the terms in (\ref{eq:main-theorem-result-A}) and (\ref{eq:main-theorem-result-Hn}),
are almost surely bounded by some constant $C_{1}$ for all $n>N_{1}$,
then there exists a constant $C_2$ such that
\begin{eqnarray}
 &  & \left|\Pi\left(B\left(\theta-\tilde{\theta}\right)\in A|X_{1},X_{2},\ldots,X_{n}\right)-\frac{P_{K}\left(A,n\right)}{P_{K}\left(H_{n},n\right)}\right|\nonumber \\
 & \le & \left|\frac{\int_{A\cap H_{n}}\prod_{i=1}^{n}\hat{w}_{i}\left(\theta\right)\pi\left(\theta\right)\diff n^{-\frac{1}{2}}Y}{\int_{H_{n}}\prod_{i=1}^{n}\hat{w}_{i}\left(\theta\right)\pi\left(\theta\right)\diff n^{-\frac{1}{2}}Y}-\frac{P_{K}\left(A,n\right)}{P_{K}\left(H_{n},n\right)}\right|\nonumber \\
 & = & \Bigg|\frac{\int_{A\cap H_{n}}\prod_{i=1}^{n}\hat{w}_{i}\left(\theta\right)\pi\left(\theta\right)\diff n^{-\frac{1}{2}}Y-P_{K}\left(A,n\right)}{\int_{H_{n}}\prod_{i=1}^{n}\hat{w}_{i}\left(\theta\right)\pi\left(\theta\right)\diff n^{-\frac{1}{2}}Y}\nonumber \\
 &  & -\frac{\left(\int_{H_{n}}\prod_{i=1}^{n}\hat{w}_{i}\left(\theta\right)\pi\left(\theta\right)\diff n^{-\frac{1}{2}}Y-P_{K}\left(H_{n},n\right)\right)P_{K}\left(A,n\right)}{\int_{H_{n}}\prod_{i=1}^{n}\hat{w}_{i}\left(\theta\right)\pi\left(\theta\right)\diff n^{-\frac{1}{2}}YP_{K}\left(H_{n},n\right)}\Bigg|\nonumber \\
 & \le & \frac{1}{\left|\int_{H_{n}}\prod_{i=1}^{n}\hat{w}_{i}\left(\theta\right)\pi\left(\theta\right)\diff n^{-\frac{1}{2}}Y\right|}\Bigg(\left|\int_{A\cap H_{n}}\prod_{i=1}^{n}\hat{w}_{i}\left(\theta\right)\pi\left(\theta\right)\diff n^{-\frac{1}{2}}Y-P_{K}\left(A,n\right)\right|\nonumber \\
 &  & +\left|\int_{H_{n}}\prod_{i=1}^{n}\hat{w}_{i}\left(\theta\right)\pi\left(\theta\right)\diff n^{-\frac{1}{2}}Y-P_{K}\left(H_{n},n\right)\right|\left|\frac{P_{K}\left(A,n\right)}{P_{K}\left(H_{n},n\right)}\right|\Bigg)\nonumber \\
 & \le & \frac{1}{C_{1}}\left(M_{1}n^{-\frac{K+2}{2}}+M_{1}n^{-\frac{K+2}{2}}\frac{C_{2}}{C_{1}}\right)=\frac{M_{1}}{C_{1}}\left(1+\frac{C_{1}}{C_{2}}\right)n^{-\frac{K+2}{2}}.\label{eq:two-quotient-close}
\end{eqnarray}
Now we find the quotient series of $P_{K}\left(A,n\right)/P_{K}\left(H_{n},n\right)$.
Let 
\[
\frac{P_{K}\left(A,n\right)}{P_{K}\left(H_{n},n\right)}=\sum_{i=0}^{\infty}\gamma_{i}\left(A,n\right)n^{-\frac{i}{2}}.
\]
Then by the product rule of series, the coefficients $\gamma_{i}\left(A,n\right)$
are determined by 
\[
\int_{A\cap H_{n}}\exp\left(-\frac{Y^{T}Y}{2}\right)\alpha_{h}\left(Y,n\right)\diff Y=\sum_{j=0}^{h}\int_{H_{n}}\exp\left(-\frac{Y^{T}Y}{2}\right)\alpha_{j}\left(Y,n\right)\diff Y\gamma_{h-j}\left(A,n\right).
\]
Through simple calculation, we find first two items of $\gamma_{i}\left(A,n\right)$
as
\begin{eqnarray*}
\gamma_{0}\left(A,n\right) & = & \frac{\rho\left(\tilde{\theta}\right)\int_{A\cap H_{n}}\exp\left(-{Y^{T}Y}/{2}\right)\diff Y}{\rho\left(\right)\int_{H_{n}}\exp\left(-{Y^{T}Y}/{2}\right)\diff Y}=\frac{\int_{A\cap H_{n}}\exp\left(-{Y^{T}Y}/{2}\right)\diff Y}{\int_{H_{n}}\exp\left(-{Y^{T}Y}/{2}\right)\diff Y}=\Phi_{p}\left(A|H_{n}\right),\\
\gamma_{1}\left(A,n\right) & = & \frac{\int_{A\cap H_{n}}\exp\left(-{Y^{T}Y}/{2}\right)\left(\delta_{1}\rho+\rho\left(\tilde{\theta}\right)\delta_{3}\tilde{l}\right)\diff Y}{\int_{H_{n}}\exp\left(-\frac{Y^{T}Y}{2}\right)\diff Y}\\
 &  & -\frac{-\int_{H_{n}}\exp\left(-{Y^{T}Y}/{2}\right)\left(\delta_{1}\rho+\rho\left(\tilde{\theta}\right)\delta_{3}\hat{l}\right)\diff Y\Phi_{p}\left(A|H_{n}\right)}{\int_{H_{n}}\exp\left(-{Y^{T}Y}/{2}\right)\diff Y}\\
 & = & \frac{\int_{A\cap H_{n}}\left(\delta_{1}\rho+\rho\left(\tilde{\theta}\right)\delta_{3}\hat{l}\right)\varphi_{p}\left(Y\right)\diff Y}{\Phi_{p}\left(H_{n}\right)}-\frac{\int_{H_{n}}\left(\delta_{1}\rho+\rho\left(\tilde{\theta}\right)\delta_{3}\hat{l}\right)\varphi_{p}\left(Y\right)\diff Y}{\Phi_{p}\left(H_{n}\right)}\Phi_{p}\left(A|H_{n}\right),
\end{eqnarray*}
where $\varphi_{p}$ is the density of  $p$-dimensional standard normal
distribution, $\Phi_{p}$ is the probability or conditional probability
of dimension $p$ standard normal distribution. By the discussion
following Lemma \ref{lem:finite-empirical-weight-moment}, we know that
all $\gamma_{i}$ are almost surely uniformly bounded for all large
$n$. Then there exists a constant $M_{2}$, such that 
\begin{equation}
\left|\frac{P_{K}\left(A,n\right)}{P_{K}\left(H_{n},n\right)}-\Phi_{p}\left(A|H_{n}\right)-\sum_{i=1}^{K}\gamma_{i}\left(A,n\right)n^{-\frac{i}{2}}\right|\le M_{2}n^{-\frac{K+1}{2}}.\label{eq:quotient-serier-approx}
\end{equation}
We combine (\ref{eq:two-quotient-close}) and (\ref{eq:quotient-serier-approx})
to get \eqref{asym-exp-poster-prob}.
\end{proof}
\end{thm}
We provide detailed proof of the this theorem in the appendix. The proof
is similar to \citet{johnson1970asymptotic}, with some modification
needed in the empirical likelihood framework. 

Let $K=2$. Then we get asymptotic normality of the posterior distribution. 
\begin{cor}[Asymptotic normality]
Use the assumption in Theorem \ref{thm:main-thm} with $K=2$, then
the posterior distribution converges in distribution to normal distribution
almost surely, that is 
\[
\left.\sqrt{n}B^T\left(\theta-\tilde{\theta}\right)\right|X_{1},\ldots,X_{n}\rightarrow N(0,I),\ascv,
\]
\end{cor}


