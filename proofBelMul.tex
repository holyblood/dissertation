\chapter{Proof for Higher-Order Properties of Bayesian Empirical Likelihood:
Multivariate Case}

In the appendix, we give the detail proof of the theorem %
\begin{comment}
asymptotic expansion theorem 
\end{comment}
. 


\section{differentiability with respect to $\theta$. }

In this section, we prove the fact that the empirical likelihood is
a smooth function of the mean parameter $\theta$. Thus we can take
arbitrary order of derivative with respect to $\theta$. We begin
the proof by the fact that the Lagrange multipliers are smooth functions
of $\theta$. 
\begin{lem}
\label{lem:mul-el-smooth-lagrange-multp}Under the assumption \enuref{multv-diff-full-rank},
$\nu\left(\theta\right)\in C^{\infty}\left(H_{n}\right)$ with probability
1 in $P_{X}^{n}$. \end{lem}
\begin{proof}
$\nu\left(\theta\right)$ is a implicit function defined by \eqref{lambda-eq}
. Let 
\begin{eqnarray*}
G_{1}\left(\nu,\theta\right) & = & \sum_{i=1}^{n}\frac{X_{i}-\theta}{1+\nu^{T}\left(X_{i}-\theta\right)},
\end{eqnarray*}
By implicit function theorem, we only need to show that $\det\left(\frac{\partial G_{1}}{\partial\nu}\right)\neq0$.
For any $1\le j,l\le p$, 
\[
\frac{\partial G_{1j}}{\partial\nu_{l}}=\frac{\partial}{\partial\nu_{l}}\sum_{i=1}^{n}\frac{X_{ij}-\theta_{j}}{1+\sum_{k=1}^{p}\nu_{k}\left(X_{ik}-\theta_{k}\right)}=-\sum_{i=1}^{n}\frac{X_{ij}-\theta_{j}}{1+\nu^{T}\left(X_{i}-\theta\right)}\frac{X_{il}-\theta_{l}}{1+\nu^{T}\left(X_{i}-\theta\right)}.
\]
Let 
\[
\Delta=\left(\frac{X_{ij}-\theta_{j}}{1+\nu^{T}\left(X_{i}-\theta\right)}\right)_{n\times p}.
\]
Then $\frac{\partial G_{1}}{\partial\nu}=\Delta^{T}\Delta$. If we
want $\det\left(\frac{\partial G_{1}}{\partial\nu}\right)\neq0$ ,
we need $\Delta$ has a full column rank. Suppose $\mathrm{rank}\left(\Delta\right)<p$,
then there is a vector $\alpha\neq0$, such that $\Delta\alpha=0$,
that is 
\begin{eqnarray*}
\sum_{j=1}^{p}\frac{X_{ij}-\theta_{j}}{1+\nu^{T}\left(X_{i}-\theta\right)}\alpha_{j} & = & 0,\\
\frac{1}{1+\nu^{T}\left(X_{i}-\theta\right)}\sum_{j=1}^{p}\left(\alpha_{j}X_{ij}-\alpha_{j}\theta_{j}\right) & = & 0.
\end{eqnarray*}
Note that $\hat{w}_{i}=\frac{1}{n\left[1+\nu^{T}\left(X_{i}-\theta\right)\right]}>0$,
so we have 
\[
\sum_{j=1}^{p}\alpha_{j}X_{ij}=\sum_{j=1}^{p}\alpha_{j}\theta_{j}.
\]
For any $1\le i\neq l\le p$, 
\[
\sum_{j=1}^{p}\alpha_{j}X_{ij}=\sum_{j=1}^{p}\alpha_{j}\theta_{j}=\sum_{j=1}^{p}\alpha_{j}X_{lj},
\]
that is 
\begin{eqnarray*}
\sum_{j=1}^{p}\left(X_{ij}-X_{lj}\right)\alpha_{j} & = & 0,\\
Z\alpha & = & 0.
\end{eqnarray*}
Contradictory with the assumption \enuref{multv-diff-full-rank}.
Hence, we have $\det\left(\frac{\partial G_{1}}{\partial\nu}\right)\neq0$,
and therefore, by implicit function theorem, $\nu$ is differentiable
with respect to $\theta$. Now we prove the existence of higher order
derivatives by induction. For $k=1$, take the derivative with respect
to $\theta_{l}$ at the both sides of \eqref{lambda-eq}, 
\[
\sum_{i=1}^{n}\sum_{s=1}^{p}\frac{\partial\nu_{s}}{\partial\theta_{l}}\frac{X_{is}-\theta_{s}}{1+\nu^{T}\left(X_{i}-\theta\right)}\frac{X_{ij}-\theta_{j}}{1+\nu^{T}\left(X_{i}-\theta\right)}=-\delta_{jl}\sum_{i=1}^{n}\frac{1}{1+\nu^{T}\left(X_{i}-\theta\right)}+\nu_{l}\sum_{i=1}^{n}\frac{X_{ij}-\theta_{j}}{\left[1+\nu^{T}\left(X_{i}-\theta\right)\right]^{2}},
\]
where $\delta_{jl}$ is Kronecker's delta, or in matrix form 
\[
\left(\Delta^{T}\Delta\right)\frac{\partial\nu}{\partial\theta}=-nI_{p}+n\Delta^{T}W\nu,
\]
where $W$ is the vector of weights $\hat{w}_{i}\left(\theta\right)$.
Hence 
\[
\frac{\partial\nu}{\partial\theta}=-n\left(\Delta^{T}\Delta\right)^{-1}+n\left(\Delta^{T}\Delta\right)^{-1}\Delta^{T}W\nu^{T}.
\]
We know both $\Delta$, $W$ and $\nu$ are differentiable with respect
to $\theta$. Suppose $k\le K$, we have 
\[
\nabla^{k}\nu=f_{k}\left(\nu\left(\theta\right)\right),
\]
where $f_{k}\in C^{\infty}$ are a bunch of smooth functions of $\nu$,
then for $k=K+1$, 
\[
\nabla^{K+1}\nu=\nabla\left(\nabla^{K}\nu\right).
\]
We know that $\nu$ is differentiable, and $f_{K}$ are smooth functions,
so the gradient above is well defined, and 
\[
\nabla^{K+1}\nu=\nabla f_{K}\left(\nu\right)\nabla\nu,
\]
are also smooth functions. By mathematical induction, we have $\nu\in C^{\infty}\left(H_{n}\right)$. 
\end{proof}
The smoothness of Lagrange multipliers leads to the smoothness of
empirical likelihood weights, and thus logarithm of empirical likelihood.
These results justify the Taylor expansion of logarithm of empirical
likelihood around the sample mean. 


\section{the behavior of the logarithm of empirical likelihood and higher
order derivatives around sample mean}

We expand the posterior around the sample mean which is indeed the
maximum of log empirical likelihood. 
\begin{lem}
\label{lem:mean-max-el}$\overline{X}$ maximizes the log empirical
likelihood $\hat{l}\left(\theta\right)$, and $\nabla\hat{l}\left(\overline{X}\right)=0$. \end{lem}
\begin{proof}
By the inequality of arithmetic mean and geometric mean, we have 
\begin{equation}
\hat{l}\left(\theta\right)=\frac{1}{n}\sum_{i=1}^{n}\log\hat{w}_{i}\left(\theta\right)=\log\left(\prod_{i=1}^{n}\hat{w}_{i}\left(\theta\right)\right)^{\frac{1}{n}}\le\log\frac{\sum_{i=1}^{n}\hat{w}_{i}\left(\theta\right)}{n}=\log\frac{1}{n},\label{eq:bound-log-el}
\end{equation}
where the equality holds if and only if all the $\hat{w}_{i}\left(\theta\right)$
are equal, i.e. , $\hat{w}_{i}\left(\theta\right)=n^{-1}.$ So $\theta=\sum_{i=1}^{n}\hat{w}_{i}\left(\theta\right)X_{i}=\sum_{i=1}^{n}n^{-1}X_{i}=\overline{X}$.
So $\hat{w}_{i}\left(\overline{X}\right)=n^{-1}$ and $\nu\left(\overline{X}\right)=0$.
By \lemref{mul-el-smooth-lagrange-multp}, $\hat{l}\left(\theta\right)$
is a smooth function of $\theta$, so at the maximal $\overline{X}$,
$\nabla\hat{l}\left(\theta\right)=0$. 
\end{proof}
Next, we prove that the higher order derivatives of log empirical
likelihood evaluated at $\overline{X}$ are solely smooth functions
of the sample central moments. Then the remainder of Taylor expansion
can be bounded by the power of $n^{-1}$. 
\begin{lem}
\label{lem:control-higher-order-derivative-l}For any $k=2,3,\ldots$
, 
\[
\nabla^{k}\hat{l}\left(\theta\right)=f_{k}\left(\frac{1}{n}\sum_{i=1}^{n}\frac{\prod_{s=1}^{l_{1}}\left(X_{ij_{s}}-\theta_{j_{s}}\right)}{\left[1+\nu^{T}\left(X_{i}-\theta\right)\right]^{t_{l_{1},1}}},\frac{1}{n}\sum_{i=1}^{n}\frac{\prod_{s=1}^{l_{1}}\left(X_{ij_{s}}-\theta_{j_{s}}\right)}{\left[1+\nu^{T}\left(X_{i}-\theta\right)\right]^{t_{l_{1},2}}},\ldots,\frac{1}{n}\sum_{i=1}^{n}\frac{\prod_{s=1}^{l_{m}}\left(X_{ij_{s}}-\theta_{j_{s}}\right)}{\left[1+\nu^{T}\left(X_{i}-\theta\right)\right]^{t_{l_{m},q}}},\nu\right),
\]
where $f_{k}$ are rational function and the denominator are only
the power of $\Delta^{T}\Delta$, $1\le l_{i}\le k$, and $l_{i}\le t_{l_{i},j}\le C_{k}<\infty$. \end{lem}
\begin{proof}
We prove this lemma use mathematical induction. When $u=2$, by \eqref{expression-fisher-inf-mat}
below, we have 
\[
\nabla^{2}\hat{l}\left(\theta\right)=f_{2}\left(\frac{1}{n}\sum_{i=1}^{n}\frac{\left(X_{ij}-\theta_{j}\right)\left(X_{is}-\theta_{s}\right)}{\left[1+\nu^{T}\left(X_{i}-\theta\right)\right]^{2}},\nu\right).
\]
Suppose for $u=k$, we have 
\[
\nabla^{k}\hat{l}\left(\theta\right)=f_{k}\left(\frac{1}{n}\sum_{i=1}^{n}\frac{\prod_{s=1}^{l_{1}}\left(X_{ij_{s}}-\theta_{j_{s}}\right)}{\left[1+\nu^{T}\left(X_{i}-\theta\right)\right]^{t_{l_{1},1}}},\frac{1}{n}\sum_{i=1}^{n}\frac{\prod_{s=1}^{l_{1}}\left(X_{ij_{s}}-\theta_{j_{s}}\right)}{\left[1+\nu^{T}\left(X_{i}-\theta\right)\right]^{t_{l_{1},2}}},\ldots,\frac{1}{n}\sum_{i=1}^{n}\frac{\prod_{s=1}^{l_{m}}\left(X_{ij_{s}}-\theta_{j_{s}}\right)}{\left[1+\nu^{T}\left(X_{i}-\theta\right)\right]^{t_{l_{m},q}}},\nu\right).
\]
When $u=k+1$, for any $v=1,2,\ldots,p$, 
\[
\frac{\partial\nabla^{k}\hat{l}\left(\theta\right)}{\partial\theta_{v}}=\nabla f_{k}\frac{\partial}{\partial\theta_{v}}\left(\frac{1}{n}\sum_{i=1}^{n}\frac{\prod_{s=1}^{l_{1}}\left(X_{ij_{s}}-\theta_{j_{s}}\right)}{\left[1+\nu^{T}\left(X_{i}-\theta\right)\right]^{t_{l_{1},1}}},\ldots,\frac{1}{n}\sum_{i=1}^{n}\frac{\prod_{s=1}^{l_{m}}\left(X_{ij_{s}}-\theta_{j_{s}}\right)}{\left[1+\nu^{T}\left(X_{i}-\theta\right)\right]^{t_{l_{m},q}}},\nu\right).
\]
For any $l$ and $t$ , 
\begin{eqnarray*}
 &  & \frac{\partial}{\partial\theta_{v}}\frac{1}{n}\sum_{i=1}^{n}\frac{\prod_{s=1}^{l}\left(X_{ij_{s}}-\theta_{j_{s}}\right)}{\left[1+\nu^{T}\left(X_{i}-\theta\right)\right]^{t}}\\
 & = & \frac{1}{n}\sum_{i=1}^{n}\frac{1}{\left[1+\nu^{T}\left(X_{i}-\theta\right)\right]^{2t}}\Bigg\{-\sum_{s=1}^{l}\delta_{j_{s}v}\prod_{r\neq s}\left(X_{ij_{r}}-\theta_{j_{r}}\right)\left[1+\nu^{T}\left(X_{i}-\theta\right)\right]^{t}-\prod_{s=1}^{l}\left(X_{ij_{s}}-\theta_{j_{s}}\right)\\
 &  & \times t\left[1+\nu^{T}\left(X_{i}-\theta\right)\right]^{t-1}\left[\sum_{h=1}^{p}\frac{\partial\nu_{h}}{\partial\theta_{t}}\left(X_{ih}-\theta_{h}\right)-\nu_{v}\right]\Bigg\}\\
 & = & -\sum_{s=1}^{l}\delta_{j_{s}v}\left\{ \frac{1}{n}\sum_{i=1}^{n}\frac{\prod_{r\neq s}\left(X_{ij_{r}}-\theta_{j_{r}}\right)}{\left[1+\nu^{T}\left(X_{i}-\theta\right)\right]^{t}}\right\} -t\sum_{h=1}^{p}\frac{\partial\nu_{h}}{\partial\theta_{t}}\left\{ \frac{1}{n}\sum_{i=1}^{n}\frac{\left(X_{ih}-\theta_{h}\right)\prod_{s=1}^{l}\left(X_{ij_{s}}-\theta_{j_{s}}\right)}{\left[1+\nu^{T}\left(X_{i}-\theta\right)\right]^{t+1}}\right\} \\
 &  & +t\nu_{v}\left\{ \frac{1}{n}\sum_{i=1}^{n}\frac{\prod_{s=1}^{l}\left(X_{ij_{s}}-\theta_{j_{s}}\right)}{\left[1+\nu^{T}\left(X_{i}-\theta\right)\right]^{t+1}}\right\} ,
\end{eqnarray*}
which is still a smooth function of $\frac{1}{n}\sum_{i=1}^{n}\frac{\prod_{s=1}^{l}\left(X_{ij_{s}}-\theta_{j_{s}}\right)}{\left[1+\nu^{T}\left(X_{i}-\theta\right)\right]^{t}}$,
and the denominator is only introduced by term $\frac{\partial\nu}{\partial\theta}$,
which contain only $\Delta^{T}\Delta$. Note that the gradient of
a rational function $f_{k}$ is still a rational function. Let $C_{k+1}=C_{k}+1$.
Then the statement holds when $u=k+1$, by mathematical induction,
it also holds for any $k$. 
\end{proof}
By \lemref{control-higher-order-derivative-l}, when evaluated at
$\theta=\overline{X}$, since $\nu\left(\overline{X}\right)=0$, then
\[
\nabla^{k}\hat{l}\left(\overline{X}\right)=f_{k}\left(\frac{1}{n}\sum_{i=1}^{n}\prod_{s=1}^{l_{1}}\left(X_{ij_{s}}-\theta_{j_{s}}\right),\ldots,\frac{1}{n}\sum_{i=1}^{n}\prod_{s=1}^{l_{m}}\left(X_{ij_{s}}-\theta_{j_{s}}\right)\right).
\]
\begin{comment}
cut paste to other place
\end{comment}
By assumption \enuref{finite-moment-sample} and SLLN, when $n\rightarrow\infty$,
\[
\frac{1}{n}\sum_{i=1}^{n}\prod_{s=1}^{l}\left(X_{ij_{s}}-\overline{X}_{j_{s}}\right)\rightarrow E\prod_{s=1}^{l}\left(X_{ij_{s}}-\theta_{j_{s},0}\right)<\infty,\ascv.
\]
\begin{comment}
not enough for higher order derivative to be finite, need some properties
for f\_k.
\end{comment}
By the smoothness of $f_{k}$ , for any $\varepsilon$, there exist
a sufficient large $N_{\left(k\right)}$, such that for any $n>N_{\left(k\right)}$,
\[
\left|\nabla^{k}\hat{l}\left(\overline{X}\right)-f_{k}\left(E\prod_{s=1}^{l}\left(X_{ij_{s}}-\theta_{j_{s},0}\right)\right)\right|<\varepsilon,\ascv.
\]
By smoothness of $\nabla^{k}\hat{l}\left(\theta\right)$, there exist
a $\delta>0$, such that for any $\left|B\left(\theta-\overline{X}\right)\right|<\delta$,
\[
\left|\nabla^{k}\hat{l}\left(\theta\right)-\nabla^{k}\hat{l}\left(\overline{X}\right)\right|<\varepsilon.
\]
Hence, 
\[
\left|\nabla^{k}\hat{l}\left(\theta\right)-f_{k}\left(E\prod_{s=1}^{l}\left(X_{ij_{s}}-\theta_{j_{s},0}\right)\right)\right|\le\left|\nabla^{k}\hat{l}\left(\overline{X}\right)-f_{k}\left(E\prod_{s=1}^{l}\left(X_{ij_{s}}-\theta_{j_{s},0}\right)\right)\right|+\left|\nabla^{k}\hat{l}\left(\theta\right)-\nabla^{k}\hat{l}\left(\overline{X}\right)\right|<2\varepsilon.
\]
Note that $f_{k}$ is a rational function and the denominator contains
only the power of $\Delta^{T}\Delta$, then there exists a constant
$M_{\left(k\right)}$, such that $\left|f_{k}\left(E\prod_{s=1}^{l}\left(X_{ij_{s}}-\theta_{j_{s},0}\right)\right)\right|<M_{\left(k\right)}.$
Therefore, 
\[
M_{\left(k\right)}-2\varepsilon<\nabla^{k}\hat{l}\left(\theta\right)<M_{\left(k\right)}+2\varepsilon.
\]
Particularly, we can compute $\nabla^{2}\hat{l}\left(\overline{X}\right)=n\left[\sum_{i=1}^{n}\left(X_{i}-\overline{X}\right)^{2}\right]^{-1}$,
which is a positive definite matrix with probability $1$ in $P_{X}^{n}$
by assumption \enuref{pd-sample-var} . 
\begin{lem}
\label{lem:near-mean-2nd-order-bound}Under the assumption \enuref{pd-sample-var},there
exists a $\delta_{1}>0$, such that 
\[
\sum_{i=1}^{n}\ln\hat{w}_{i}\left(\theta\right)-\sum_{i=1}^{n}\ln\hat{w}_{i}\left(\overline{X}\right)\le-\frac{1}{4}Y^{T}Y,
\]
for any $\theta\in\left\{ \left|B\left(\theta-\overline{X}\right)\right|\le\delta_{1}\right\} \cap H_{n}$
. \end{lem}
\begin{proof}
By Taylor expansion, 
\begin{eqnarray*}
\frac{1}{n}\left(\sum_{i=1}^{n}\ln\hat{w}_{i}\left(\theta\right)-\sum_{i=1}^{n}\ln\hat{w}_{i}\left(\overline{X}\right)\right) & = & \left(\nabla\hat{l}\left(\overline{X}\right)\right)^{T}\left(\theta-\overline{X}\right)+\frac{1}{2}\left(\theta-\overline{X}\right)^{T}\nabla^{2}\hat{l}\left(\theta^{*}\right)\left(\theta-\overline{X}\right),
\end{eqnarray*}
where $\left|\theta^{*}-\overline{X}\right|\le\left|\theta-\overline{X}\right|$.
By \lemref{mean-max-el}, the first term in above equation is zero.
By \lemref{control-higher-order-derivative-l}, we know that $\nabla^{2}\hat{l}\left(\theta\right)$
is a continuous function in $\theta$. There exists a $\delta_{1}$,
such that for any $\left|B\left(\theta^{*}-\overline{X}\right)\right|\le\left|B\left(\theta-\overline{X}\right)\right|<\delta_{1}$,
\[
\left|\left(\theta-\overline{X}\right)^{T}\nabla^{2}\hat{l}\left(\theta^{*}\right)\left(\theta-\overline{X}\right)+\left|B\left(\theta-\overline{X}\right)\right|^{2}\right|<\frac{1}{2}\left|B\left(\theta-\overline{X}\right)\right|^{2},
\]
hence $\left(\theta-\overline{X}\right)^{T}\nabla^{2}\hat{l}\left(\theta^{*}\right)\left(\theta-\overline{X}\right)<-2^{-1}\left|B\left(\theta-\overline{X}\right)\right|^{2}$.
Therefore, 
\[
\sum_{i=1}^{n}\ln\hat{w}_{i}\left(\theta\right)-\sum_{i=1}^{n}\ln\hat{w}_{i}\left(\overline{X}\right)<\frac{1}{2}\times\frac{1}{2}\left|\sqrt{n}B\left(\theta-\overline{X}\right)\right|^{2}=\frac{1}{4}Y^{T}Y.
\]

\end{proof}

\section{bell shape of the logarithm of empirical likelihood}

If we want the expansion of posterior is uniform on the parameter
space, we do not only need the Taylor expansion around the sample
mean, but also require controlling the value of posterior in the region
far from sample mean. This can be achieved by showing empirical likelihood
have ``bell'' shape around the sample mean. 
\begin{lem}
\label{lem:second-order-der-neg-def}Under the assumption \enuref{multv-diff-full-rank}
and \enuref{finite-moment-sample}, $\frac{\partial^{2}\hat{l}\left(\theta\right)}{\partial\theta\partial\theta^{T}}$
is negative definite almost surely in $P_{X}^{n}$. \end{lem}
\begin{proof}
For any $j=1,2,\ldots,p$, take the first derivative of logarithm
of empirical likelihood,
\begin{eqnarray*}
\frac{\partial\hat{l}_{n}\left(\theta\right)}{\partial\theta_{j}} & = & \frac{1}{n}\frac{\partial}{\partial\theta_{j}}\sum_{i=1}^{n}\left[-\ln\left(1+\sum_{k=1}^{p}\nu_{k}\left(\theta\right)\left(X_{ik}-\theta_{k}\right)\right)\right]\\
 & = & \sum_{i=1}^{n}\sum_{k=1}^{p}\frac{\partial\nu_{k}}{\partial\theta_{j}}\frac{X_{ik}-\theta_{k}}{n\left[1+\nu^{T}\left(X_{i}-\theta\right)\right]}+\sum_{i=1}^{n}\frac{1}{n\left[1+\nu^{T}\left(X_{i}-\theta\right)\right]}\nu_{j}=\nu_{j}.
\end{eqnarray*}
Then the Hessian of logarithm of empirical likelihood is 
\[
\frac{\partial^{2}\hat{l}\left(\theta\right)}{\partial\theta^{T}\partial\theta}=\frac{\partial\nu}{\partial\theta}=-n\left(\Delta^{T}\Delta\right)^{-1}+n\left(\Delta^{T}\Delta\right)^{-1}\Delta^{T}W\nu^{T}.
\]
Note that 
\[
\Delta\nu=\left(\frac{\nu^{T}\left(X_{i}-\theta\right)}{1+\nu^{T}\left(X_{i}-\theta\right)}\right)_{i=1}^{n}=\left(1-\frac{1}{1+\lambda^{T}\left(X_{i}-\theta\right)}\right)_{i=1}^{n}=1_{n\times1}-nW,
\]
where $1_{n\times1}$ is a column vector with all entries are equal
to 1, so 
\[
W=\frac{1}{n}\left(1_{n\times1}-\Delta\nu\right).
\]
 Replace $W$ in Hessian, 
\begin{eqnarray*}
\frac{\partial^{2}\hat{l}\left(\theta\right)}{\partial\theta^{T}\partial\theta} & = & n\left[-\left(\Delta^{T}\Delta\right)^{-1}+\left(\Delta^{T}\Delta\right)^{-1}\Delta^{T}\frac{1}{n}\left(1_{n\times1}-\Delta\nu\right)\nu^{T}\right]\\
 & = & -n\left(\Delta^{T}\Delta\right)^{-1}+\left(\Delta^{T}\Delta\right)^{-1}\left(\Delta^{T}1_{n\times1}\right)-\left(\Delta^{T}\Delta\right)^{-1}\Delta^{T}\Delta\nu\nu^{T}.
\end{eqnarray*}
Also note that 
\[
\Delta^{T}1_{n\times1}=\left(\sum_{i=1}^{n}\frac{X_{ij}-\theta_{j}}{1+\nu^{T}\left(X_{i}-\theta\right)}\right)_{j=1}^{n}=0.
\]
Therefore,
\begin{equation}
\frac{\partial^{2}\hat{l}\left(\theta\right)}{\partial\theta^{T}\partial\theta}=-n\left(\Delta^{T}\Delta\right)^{-1}-\nu\nu^{T},\label{eq:expression-fisher-inf-mat}
\end{equation}
is obviously negative definite since in proof of \lemref{mul-el-smooth-lagrange-multp},
we know that $\Delta$ has a full column rank. Particularly when $\theta=\overline{X}$,
$\frac{\partial^{2}\hat{l}\left(\overline{X}\right)}{\partial\theta^{T}\partial\theta}=-\left[\frac{1}{n}\sum_{i=1}^{n}\left(X_{i}-\overline{X}\right)^{2}\right]^{-1}$. 
\end{proof}
Now we can control the tail part by the following lemma.
\begin{lem}
\label{lem:exponential-decay-tail}Under the assumption \enuref{multv-diff-full-rank},
for any $\delta>0$, there exists an $\varepsilon>0$, $N_{2}$, and
$D_{1}\subset\mathbb{R}^{p}$, $P_{X}^{n}\left(D_{1}\right)=1$, such
that 
\[
\hat{l}\left(\theta\right)-\hat{l}\left(\overline{X}\right)\le-\varepsilon,\ascv,
\]
for any $\left|B\left(\theta-\overline{X}\right)\right|\ge\delta$
and $\theta\in H_{n}$. \end{lem}
\begin{proof}
In the proof of \lemref{second-order-der-neg-def}, we know that $\nu\left(\overline{X}\right)=0$,
thus $\hat{l}\left(\overline{X}\right)=-\log n$ and score function
$\hat{l}'\left(\overline{X}\right)=\nu\left(\overline{X}\right)=0$.
By strictly convexity of log empirical likelihood from \lemref{second-order-der-neg-def},
we know that $\overline{X}$ is the unique maximal. Therefore, for
any $\theta\neq\overline{X}$, $\hat{l}\left(\theta\right)<\hat{l}\left(\overline{X}\right)$.
Note that the set 
\[
\left\{ \theta|\left|B\left(\theta-\overline{X}\right)\right|\ge\delta\right\} \cap H_{n}
\]
is a compact set, and $\hat{l}\left(\theta\right)$ is a continuous
function, then there exists a $\theta^{*}\in\left\{ \theta|\left|B\left(\theta-\overline{X}\right)\right|\ge\delta\right\} \cap H_{n}$
, such that for any $\theta\in\left\{ \theta|\left|B\left(\theta-\overline{X}\right)\right|\ge\delta\right\} \cap H_{n}$,
\[
\hat{l}\left(\theta\right)\le\hat{l}\left(\theta^{*}\right).
\]
Therefore, $\hat{l}\left(\theta\right)\le\hat{l}\left(\theta^{*}\right)<\hat{l}\left(\overline{X}\right)$,
\[
\hat{l}\left(\theta\right)-\hat{l}\left(\overline{X}\right)\le\hat{l}\left(\theta^{*}\right)-\hat{l}\left(\overline{X}\right)<0.
\]
Let $\varepsilon=\frac{1}{2}\left(\hat{l}\left(\overline{X}\right)-\hat{l}\left(\theta^{*}\right)\right)$,
then we have 
\[
\hat{l}\left(\theta\right)-\hat{l}\left(\overline{X}\right)\le\hat{l}\left(\theta^{*}\right)-\hat{l}\left(\overline{X}\right)<\varepsilon.
\]

\end{proof}

\section{expansion near the sample mean}

After we control the tail of the posterior, the remaining work are
solely based on smoothness of the log empirical likelihood and prior.
Particularly the form of expansion polynomials are determined by the
product of Taylor expansion of empirical likelihood and prior at sample
mean. 
\begin{lem}
\label{lem:central-expansion-llik}%
\begin{comment}
need condition
\end{comment}
There exist a $\delta_{2}$, a constant $M_{1}$ and $N_{1}$, such
that 
\[
\left|\int_{\left\{ B\left(\theta-\overline{X}\right)\le\delta_{2}\right\} \cap H_{n}}\exp\left(-\frac{1}{2}Y^{T}Y+\sum_{k=3}^{K+3}\delta_{k}\hat{l}n^{-\frac{k-2}{2}}\right)-\prod_{i=1}^{n}\frac{\hat{w}_{i}\left(\theta\right)}{\hat{w}_{i}\left(\overline{X}\right)}\diff n^{-\frac{1}{2}}Y\right|\le M_{1}n^{-\frac{1}{2}\left(K+3\right)},\:\ascv.
\]
\end{lem}
\begin{proof}
First, we can choose $\delta_{2}$ small enough so that $\left\{ B\left(\theta-\overline{X}\right)\le\delta_{2}\right\} \subset H_{n}$.
\begin{eqnarray*}
 &  & \int_{\left\{ B\left(\theta-\overline{X}\right)\le\delta_{2}\right\} \cap H_{n}}\exp\left(-\frac{1}{2}Y^{T}Y+\sum_{k=3}^{K+3}\delta_{k}\hat{l}n^{-\frac{k-2}{2}}\right)-\prod_{i=1}^{n}\frac{\hat{w}_{i}\left(\theta\right)}{\hat{w}_{i}\left(\overline{X}\right)}\diff n^{-\frac{1}{2}}Y\\
 & = & \int_{\left\{ B\left(\theta-\overline{X}\right)\le\delta_{2}\right\} }\exp\left(\sum_{i=1}^{n}\ln\hat{w}_{i}\left(\theta\right)-\sum_{i=1}^{n}\ln\hat{w}_{i}\left(\overline{X}\right)\right)\\
 &  & \left[\exp\left(\left(-\frac{1}{2}Y^{T}Y+\sum_{k=3}^{K+3}\delta_{k}\hat{l}n^{-\frac{k-2}{2}}\right)-\sum_{i=1}^{n}\left(\ln\hat{w}_{i}\left(\theta\right)-\ln\hat{w}_{i}\left(\overline{X}\right)\right)\right)-1\right]\diff n^{-\frac{1}{2}}Y.
\end{eqnarray*}
By \lemref{near-mean-2nd-order-bound}, and Taylor expansion, the
above equation can be bounded by 
\begin{eqnarray*}
 &  & \int_{\left\{ \left|B\left(\theta-\overline{X}\right)\right|\le\delta_{2}\right\} }\exp\left(-\frac{Y^{T}Y}{4}\right)\left|\exp\left(-\frac{n^{-\frac{K+2}{2}}}{\left(K+4\right)!}\left[\left(\theta-\overline{X}\right)\nabla\right]^{K+4}\hat{l}\left(\theta^{*}\right)\right)-1\right|\diff n^{-\frac{1}{2}}Y\\
 & = & n^{-\frac{1}{2}}\int_{\left\{ \sqrt{Y^{T}Y}\le\delta_{2}\sqrt{n}\right\} }\exp\left(-\frac{Y^{T}Y}{4}\right)\left|\exp\left(-\frac{n^{-\frac{K+2}{2}}}{\left(K+4\right)!}\left(Y^{T}B^{-T}\nabla\right)^{K+4}\hat{l}\left(\theta^{*}\right)\right)-1\right|\diff Y.
\end{eqnarray*}
By \lemref{control-higher-order-derivative-l}, we know that $\nabla^{K+4}\hat{l}\left(\theta^{*}\right)$
are bounded by some constants. Note that for any $Y$, and any vector
$a$, there exists a constant $C_{1}$, such that 
\begin{equation}
\left(Y^{T}a\right)^{2}\le C_{1}\left(Y^{T}Y\right),\label{eq:average-inequality}
\end{equation}
so we can bounded the above equation by 
\[
n^{-\frac{1}{2}}\int_{\left\{ \sqrt{Y^{T}Y}\le\delta_{2}\sqrt{n}\right\} }\exp\left(-\frac{Y^{T}Y}{4}\right)\left(\exp\left(-n^{-\frac{K+2}{2}}C_{3}\left(Y^{T}Y\right)^{\frac{K+4}{2}}\right)-1\right)\diff Y,
\]
where $C_{3}=C_{1}^{\frac{K+4}{2}}$. Denote the above integrand to
be $F\left(Y,\sqrt{n}\right)$. We can the inequality in this lemma
if we can prove the following 
\begin{equation}
\lim_{n\rightarrow\infty}\frac{n^{-\frac{1}{2}}\int_{\left\{ \sqrt{Y^{T}Y}\le\delta_{2}\sqrt{n}\right\} }F\left(Y,\sqrt{n}\right)\diff Y}{n^{-\frac{1}{2}\left(K+3\right)}}=C_{2},\label{eq:lim-rhs-vs-lfs-inequality}
\end{equation}
for some constant $C_{2}<\infty$. Let $t=\sqrt{n}$, and relax $t\in\mathbb{R}^{+}$.
The the above formula can be written as 
\[
\frac{\int_{\left\{ \sqrt{Y^{T}Y}\le\delta_{2}t\right\} }F\left(Y,t\right)\diff Y}{t^{-K-2}}.
\]
Take the derivatives of both numerator and denominator with respect
to $t$. For the denominator $\left(t^{-K-2}\right)'=-\left(K+2\right)t^{-K-3}.$
For the numerator, we change to $n$ dimensional spherical coordinate
system, 
\begin{eqnarray*}
Y_{1} & = & r\cos\left(\varphi_{1}\right),\\
Y_{2} & = & r\sin\left(\varphi_{1}\right)\cos\left(\varphi_{2}\right),\\
 & \vdots\\
Y_{p} & = & r\sin\left(\varphi_{1}\right)\sin\left(\varphi_{2}\right)\cdots\sin\left(\varphi_{p-1}\right).
\end{eqnarray*}
So the numerator can be written as 
\begin{eqnarray*}
 &  & \int_{\left\{ r\le\delta_{2}t\right\} }\exp\left(-\frac{r^{2}}{4}\right)\left(\exp\left(-C_{3}t^{-K-2}r^{K+4}\right)-1\right)\\
 &  & \times r^{p-1}\sin^{p-2}\left(\varphi_{1}\right)\sin^{p-3}\left(\varphi_{2}\right)\cdots\sin\left(\varphi_{p-2}\right)\diff r\diff\varphi_{1}\cdots\diff\varphi_{p-1}\\
 & = & \int_{0}^{2\pi}\diff\varphi_{p-1}\int_{0}^{\pi}\sin^{p-2}\left(\varphi_{1}\right)\diff\varphi_{1}\cdots\int_{0}^{\pi}\sin\left(\varphi_{p-2}\right)\diff\varphi_{p-2}\\
 &  & \times\int_{0}^{\delta_{2}t}\exp\left(-\frac{r^{2}}{4}\right)\left(\exp\left(-C_{3}t^{-K-4}r^{K+4}\right)-1\right)r^{p-1}\diff r\\
 & \le & 2\pi\left(\pi\right)^{p-2}\int_{0}^{\delta_{2}t}\exp\left(-\frac{r^{2}}{4}\right)\left(\exp\left(-C_{3}t^{-K-2}r^{K+4}\right)-1\right)r^{p-1}\diff r\\
 & = & 2\pi^{p-1}\int_{0}^{\delta_{2}t}\exp\left(-\frac{r^{2}}{4}\right)\left(\exp\left(-C_{3}t^{-K-2}r^{K+4}\right)-1\right)r^{p-1}\diff r.
\end{eqnarray*}
 
\begin{eqnarray*}
 &  & \frac{\diff}{\diff t}\int_{0}^{\delta_{2}t}\exp\left(-\frac{r^{2}}{4}\right)\left(\exp\left(-C_{3}t^{-K-2}r^{K+4}\right)-1\right)r^{p-1}\diff r\\
 & = & \int_{0}^{\delta_{2}t}-\left(K+2\right)t^{-K-3}\left(-C_{3}r^{K+4}\right)\exp\left(-\frac{r^{2}}{4}\right)\exp\left(-C_{3}t^{-K-2}r^{K+4}\right)r^{p-1}\diff r\\
 &  & +\exp\left(-\frac{\delta_{2}^{2}t^{2}}{4}\right)\left(\exp\left(-C_{3}t^{-K-2}\left(\delta_{2}t\right)^{K+4}\right)-1\right)\\
 & \le & C_{3}\left(K+2\right)t^{-K-3}\int_{0}^{\delta_{2}t}r^{K+p+3}\exp\int_{A\cap H_{n}}\exp\left(-\frac{Y^{T}Y}{4}\right)\left|\alpha_{h}\left(Y,n\right)\right|\diff Y\\
 &  & \left(-\left(\frac{1}{4}-C_{3}t^{-K-2}r^{K+2}\right)r^{2}\right)\diff r+\exp\left(-\left(\frac{1}{4}-C_{3}\delta_{2}^{K+2}\right)\delta_{2}^{2}t^{2}\right)-\exp\left(-\frac{\delta_{2}^{2}t^{2}}{4}\right).
\end{eqnarray*}
If 
\[
\delta_{2}<\sqrt[K+2]{\frac{1}{4C_{3}}},
\]
then 
\[
\frac{1}{4}-C_{3}\delta_{2}^{K+2}>0,
\]
hence 
\[
\lim_{t\rightarrow+\infty}\frac{\exp\left(-\left(\frac{1}{4}-C_{3}\delta_{2}^{K+2}\right)\delta_{2}^{2}t^{2}\right)-\exp\left(-\frac{\delta_{2}^{2}t^{2}}{4}\right)}{-\left(K+2\right)t^{-K-3}}=0.
\]
For the first term in derivative of numerator, we have for any $t>0$.
\begin{eqnarray*}
 &  & \int_{0}^{\delta_{2}t}r^{K+p+3}\exp\left(-\left(\frac{1}{4}-C_{3}t^{-K-2}r^{K+2}\right)r^{2}\right)\diff r\\
 & \le & \int_{0}^{\delta_{2}t}r^{K+p+3}\exp\left(-\left(\frac{1}{4}-C_{3}\delta_{2}^{K+2}\right)r^{2}\right)\diff r\\
 & \le & \int_{\mathbb{R}}\left|r\right|^{K+p+3}\exp\left(-\left(-\frac{1}{4}-C_{3}\delta_{2}^{K+2}\right)r^{2}\right)\diff r<+\infty.
\end{eqnarray*}
Therefore, there exists a constant $C_{2}$, such that 
\begin{eqnarray*}
 &  & \lim_{t\rightarrow+\infty}\frac{C_{3}\left(K+2\right)t^{-K-3}\int_{0}^{\delta_{2}t}r^{K+p+3}\exp\left(-\left(\frac{1}{4}-C_{3}t^{-K-2}r^{K+2}\right)r^{2}\right)\diff r}{-\left(K+2\right)t^{-K-3}}\\
 & = & -C_{3}\lim_{t\rightarrow+\infty}\int_{0}^{\delta_{2}t}r^{K+p+3}\exp\left(-\left(\frac{1}{4}-C_{3}t^{-K-2}r^{K+2}\right)r^{2}\right)\diff r=C_{2}.
\end{eqnarray*}
By L'Hostiple's rule, we have \eqref{lim-rhs-vs-lfs-inequality} holds,
and therefore, the lemma holds.\end{proof}
\begin{lem}
\label{lem:central-expansion-post-prod}%
\begin{comment}
need change the subscript of N, M, C
\end{comment}
Under the assumption \enuref{multv-diff-full-rank}, \enuref{finite-moment-sample}
and \enuref{smooth-piror}, there exist a $\delta_{2}>0$, a constant
$M_{2}$ and $N_{2}$, such that 
\begin{equation}
\left|\int_{\left\{ \left|B\left(\theta-\overline{X}\right)\right|\le\delta\right\} \cap H_{n}}\exp\left(-\frac{1}{2}Y^{T}Y+\sum_{k=3}^{K+3}\delta_{k}\hat{l}n^{-\frac{k-2}{2}}\right)\rho_{K}\left(\theta\right)-\prod_{i=1}^{n}\frac{\hat{w}_{i}\left(\theta\right)}{\hat{w}_{i}\left(\overline{X}\right)}\rho\left(\theta\right)\diff n^{-\frac{1}{2}}Y\right|\le M_{2}n^{-\frac{1}{2}\left(K+2\right)},\:\ascv.\label{eq:central-exp-post}
\end{equation}
\end{lem}
\begin{proof}
Apply Taylor expansion to $\hat{l}\left(\theta\right)$ around $\overline{X}$,
for any $\theta\in H_{n}$, there exists a $\theta^{*}$satisfies
$\left|B\left(\theta^{*}-\overline{X}\right)\right|\le\left|B\left(\theta-\overline{X}\right)\right|$,
such that, 
\begin{eqnarray*}
\hat{l}\left(\theta\right) & = & \hat{l}\left(\overline{X}\right)+\nabla\hat{l}\left(\overline{X}\right)\left(\theta-\overline{X}\right)+\frac{1}{2}\left(\theta-\overline{X}\right)^{T}\frac{\partial^{2}\hat{l}\left(\overline{X}\right)}{\partial\theta^{T}\partial\theta}\left(\theta-\overline{X}\right)+\sum_{k=3}^{K+3}\frac{1}{k!}\left[\left(\theta-\overline{X}\right)^{T}\nabla\right]^{k}\hat{l}\left(\overline{X}\right)\\
 &  & +\frac{1}{\left(K+4\right)!}\left[\left(\theta-\overline{X}\right)\nabla\right]^{K+4}\hat{l}\left(\theta^{*}\right)\\
 & = & \hat{l}\left(\overline{X}\right)-\frac{1}{2}Y^{T}Yn^{-1}+\sum_{k=3}^{K+3}\delta_{k}\hat{l}n^{-\frac{k}{2}}++\frac{1}{\left(K+4\right)!}\left[\left(\theta-\overline{X}\right)\nabla\right]^{K+4}\hat{l}\left(\theta^{*}\right).
\end{eqnarray*}
Now we have 
\begin{eqnarray*}
 &  & \left|\exp\left(-\frac{1}{2}Y^{T}Y+\sum_{k=3}^{K+3}\delta_{k}\hat{l}n^{-\frac{k-2}{2}}\right)\rho_{K}\left(\theta\right)-\prod_{i=1}^{n}\frac{\hat{w}_{i}\left(\theta\right)}{\hat{w}_{i}\left(\overline{X}\right)}\rho\left(\theta\right)\right|\\
 &  & \le\left|\exp\left(-\frac{1}{2}Y^{T}Y+\sum_{k=3}^{K+3}\delta_{k}\hat{l}n^{-\frac{k-2}{2}}\right)\rho_{K}\left(\theta\right)-\prod_{i=1}^{n}\frac{\hat{w}_{i}\left(\theta\right)}{\hat{w}_{i}\left(\overline{X}\right)}\rho_{K}\left(\theta\right)\right|\\
 &  & +\left|\prod_{i=1}^{n}\frac{\hat{w}_{i}\left(\theta\right)}{\hat{w}_{i}\left(\overline{X}\right)}\rho_{K}\left(\theta\right)-\prod_{i=1}^{n}\frac{\hat{w}_{i}\left(\theta\right)}{\hat{w}_{i}\left(\overline{X}\right)}\rho\left(\theta\right)\right|\\
 &  & \le\left|\rho_{K}\left(\theta\right)\right|\exp\left(n\left(\hat{l}\left(\theta\right)-\hat{l}\left(\overline{X}\right)\right)\right)\left|\exp\left(n\left(\hat{l}\left(\overline{X}\right)-\frac{1}{2}Y^{T}Yn^{-1}+\sum_{k=3}^{K+3}\delta_{k}\hat{l}n^{-\frac{k-2}{2}}-\hat{l}\left(\theta\right)\right)\right)-1\right|\\
 &  & +\exp\left(n\left(\hat{l}\left(\theta\right)-\hat{l}\left(\overline{X}\right)\right)\right)\left|\rho_{K}\left(\theta\right)-\rho\left(\theta\right)\right|.
\end{eqnarray*}
For the first part of the above formula, by \lemref{central-expansion-llik},
we have it to be bounded by 
\[
\max_{\theta\in\left\{ \left\{ B\left(\theta-\overline{X}\right)\le\delta_{2}\right\} \cap H_{n}\right\} }\rho_{K}\left(\theta\right)M_{1}n^{-\frac{1}{2}\left(K+3\right)}.
\]
For the second part, by \lemref{near-mean-2nd-order-bound}, Taylor
expansion in $\rho\left(\theta\right)$, and \eqref{average-inequality},
we have the upper bound to be 
\begin{eqnarray*}
 &  & \int_{\left\{ Y^{T}Y\le\delta_{2}^{2}n\right\} }\exp\left(-\frac{Y^{T}Y}{4}\right)\frac{n^{-\frac{K+1}{2}}}{\left(K+1\right)!}\left(Y^{T}B^{-T}\nabla\right)^{K+1}\rho\left(\theta^{*}\right)\diff n^{-\frac{1}{2}}Y\\
 & \le & \frac{1}{\left(K+1\right)!}\max_{\theta\in\left\{ \left\{ B\left(\theta-\overline{X}\right)\le\delta_{2}\right\} \cap H_{n}\right\} }\left|\nabla^{K+1}\rho\left(\theta^{*}\right)\right|\int_{\left\{ Y^{T}Y\le\delta_{2}^{2}n\right\} }\exp\left(-\frac{Y^{T}Y}{4}\right)C_{1}^{\frac{K+1}{2}}\left(Y^{T}Y\right)^{\frac{K+1}{2}}\diff Yn^{-\frac{K+2}{2}}\\
 & \le & \frac{C_{1}^{\frac{K+1}{2}}}{\left(K+1\right)!}\max_{\theta\in\left\{ \left\{ B\left(\theta-\overline{X}\right)\le\delta_{2}\right\} \cap H_{n}\right\} }\left|\nabla^{K+1}\rho\left(\theta\right)\right|\int_{\mathbb{R}^{p}}\exp\left(-\frac{Y^{T}Y}{4}\right)\left(Y^{T}Y\right)^{\frac{K+1}{2}}\diff Yn^{-\frac{K+2}{2}}\\
 & = & \frac{C_{1}^{\frac{K+1}{2}}}{\left(K+1\right)!}\max_{\theta\in\left\{ \left\{ B\left(\theta-\overline{X}\right)\le\delta_{2}\right\} \cap H_{n}\right\} }\left|\nabla^{K+1}\rho\left(\theta\right)\right|2\pi^{p-1}\int_{0}^{\infty}\exp\left(-\frac{r^{2}}{4}\right)r^{K+1}r^{p-1}\diff rn^{-\frac{K+2}{2}}\\
 & = & \left[\frac{2\pi^{p-1}C_{1}^{\frac{K+1}{2}}}{\left(K+1\right)!}\max_{\theta\in\left\{ B\left(\theta-\overline{X}\right)\le\delta_{2}\right\} \cap H_{n}}\left|\nabla^{K+1}\rho\left(\theta\right)\right|\int_{0}^{\infty}r^{p+K}\exp\left(-\frac{r^{2}}{4}\right)\diff r\right]n^{-\frac{K+2}{2}}
\end{eqnarray*}
Since $p\ge1$, we have \eqref{central-exp-post} holds.
\end{proof}

\section{proof of the main theorem}

We first intuitively derive %
\begin{comment}
add expansion polynomial
\end{comment}
. First, we expand 
\begin{eqnarray*}
 &  & \exp\left(\sum_{k=3}^{K+3}\delta_{k}\hat{l}n^{-\frac{k-2}{2}}\right)\\
 & = & \sum_{i=0}^{K+1}\frac{1}{i!}\left(\sum_{k=3}^{K+3}\delta_{k}\hat{l}n^{-\frac{k-2}{2}}\right)^{i}\\
 & = & 1+\sum_{i=1}^{K+1}\frac{1}{i!}\sum_{\sum_{u=3}^{K+3}m_{u,i}=i}\binom{i}{m_{3,i},m_{4,i},\ldots,m_{K+3,i}}\prod_{u=3}^{K+3}\left(\delta_{u}\hat{l}\right)^{m_{u,i}}n^{-\frac{1}{2}\sum_{u=3}^{K+3}m_{u,i}\left(u-2\right)}.
\end{eqnarray*}
Then we product the above expansion by $\rho_{K}$, 
\begin{eqnarray*}
 &  & \exp\left(\sum_{k=3}^{K+3}\frac{1}{k!}\delta_{k}\hat{l}n^{-\frac{k-2}{2}}\right)\rho_{K}\left(\theta\right)\\
 & = & \left[1+\sum_{i=1}^{K+1}\frac{1}{i!}\sum_{\sum_{u=3}^{K+3}m_{u,i}=i}\binom{i}{m_{3,i},m_{4,i},\ldots,m_{K+3,i}}\prod_{u=3}^{K+3}\left(\delta_{u}\hat{l}\right)^{m_{u,i}}n^{-\frac{1}{2}\sum_{u=3}^{K+3}m_{u,i}\left(u-2\right)}\right]\\
 &  & \left(\rho\left(\overline{X}\right)+\sum_{j=1}^{K}\delta_{j}\rho n^{-\frac{j}{2}}\right)\\
 & = & \rho\left(\overline{X}\right)+\sum_{j=1}^{K}\delta_{j}\rho n^{-\frac{j}{2}}+\rho\left(\overline{X}\right)\sum_{i=1}^{K+1}\frac{1}{i!}\\
 &  & \sum_{\sum_{u=3}^{K+3}m_{u,i}=i}\binom{i}{m_{3,i},m_{4,i},\ldots,m_{K+3,i}}\prod_{u=3}^{K+3}\left(\delta_{u}\hat{l}\right)^{m_{u,i}}n^{-\frac{1}{2}\sum_{u=3}^{K+3}m_{u,i}\left(u-2\right)}\\
 &  & +\left[\sum_{i=1}^{K+1}\frac{1}{i!}\sum_{\sum_{u=3}^{K+3}m_{u,i}=i}\binom{i}{m_{3,i},m_{4,i},\ldots,m_{K+3,i}}\prod_{u=3}^{K+3}\left(\delta_{u}\hat{l}\right)^{m_{u,i}}n^{-\frac{1}{2}\sum_{u=3}^{K+3}m_{u,i}\left(u-2\right)}\right]\sum_{j=1}^{K}\delta_{j}\rho n^{-\frac{j}{2}}.
\end{eqnarray*}
For the third term in above equation, we change the summation index.
Let $\sum_{u=3}^{K+3}m_{u,i}\left(u-2\right)=h$. Note that for any
$\sum_{u=3}^{K+3}m_{u,i}=i$, $i\le h\le i\left(K+1\right)$, $h/\left(K+1\right)\le i\le h$.
Thus the third term summation can be rearranged as 
\[
\sum_{h=1}^{\left(K+1\right)^{2}}\left[\rho\left(\overline{X}\right)\sum_{\frac{h}{K+1}\le i\le h}\frac{1}{i!}\sum_{I_{i,h}}\binom{i}{m_{3,i},m_{4,i},\ldots,m_{K+3,i}}\prod_{u=3}^{K+3}\left(\delta_{u}\hat{l}\right)^{m_{u,i}}\right]n^{-\frac{h}{2}}.
\]
Similarly for the fourth term, let $\sum_{u=3}^{K+3}m_{u,i}\left(u-2\right)+j=h$,
then the summation can be rearranged as 
\[
\sum_{h=2}^{\left(K+1\right)^{2}+K}\left[\sum_{j=1}^{h-1}\delta_{j}\rho\sum_{\frac{h-j}{K+1}\le i\le h-j}\frac{1}{i!}\sum_{I_{i,h-j}}\binom{i}{m_{3,i},m_{4,i},\ldots,m_{K+3,i}}\prod_{u=3}^{K+3}\left(\delta_{u}\hat{l}\right)^{m_{u,i}}\right]n^{-\frac{h}{2}}.
\]
We collect the same order term of $n$, and denote the summation of
all the terms with order higher than $K$ to be $R_{K}\left(Y\right)$,
then we get the product as 
\begin{eqnarray*}
 &  & \rho\left(\overline{X}\right)+\left(\delta_{1}\rho+\rho\left(\overline{X}\right)\delta_{3}\hat{l}\right)n^{-\frac{1}{2}}\\
 &  & +\sum_{h=2}^{K}\left[\delta_{h}\rho+\sum_{j=0}^{h-1}\delta_{j}\rho\sum_{\frac{h-j}{K+1}\le i\le h-j}\frac{1}{i!}\sum_{I_{i,h-j}}\binom{i}{m_{3,i},m_{4,i},\ldots,m_{K+3,i}}\prod_{u=3}^{K+3}\left(\delta_{u}\hat{l}\right)^{m_{u,i}}\right]n^{-\frac{h}{2}}+R_{K}\left(Y\right).
\end{eqnarray*}
 Integral over any Borel set $A\cap H_{n}$, we can get the polynomial
$P_{K}\left(A,n\right)$. Now we can prove the main theorem %
\begin{comment}
add ref to main theorem
\end{comment}
. 
\begin{proof}
Let $A_{1}=\left\{ \left|Y\right|_{2}\ge\delta_{2}\sqrt{n}\right\} $
and $A_{2}=\left\{ \left|Y\right|_{2}<\delta_{2}\sqrt{n}\right\} $.
Then 
\begin{eqnarray*}
 &  & \left|\int_{A\cap H_{n}}\prod_{i=1}^{n}\frac{\hat{w}_{i}\left(\theta\right)}{\hat{w}_{i}\left(\overline{X}\right)}\rho\left(\theta\right)\diff n^{-\frac{1}{2}}Y-P_{K}\left(A,n\right)\right|\\
 & = & \left|\int_{A\cap H_{n}}\prod_{i=1}^{n}\frac{\hat{w}_{i}\left(\theta\right)}{\hat{w}_{i}\left(\overline{X}\right)}\rho\left(\theta\right)-\exp\left(-\frac{Y^{T}Y}{2}\right)\sum_{h=0}^{K}\alpha_{h}\left(Y,n\right)n^{-\frac{h}{2}}\diff n^{-\frac{1}{2}}Y\right|\\
 & \le & \left|\int_{A\cap H_{n}\cap A_{1}}\prod_{i=1}^{n}\frac{\hat{w}_{i}\left(\theta\right)}{\hat{w}_{i}\left(\overline{X}\right)}\rho\left(\theta\right)-\exp\left(-\frac{Y^{T}Y}{2}\right)\sum_{h=0}^{K}\alpha_{h}\left(Y,n\right)n^{-\frac{h}{2}}\diff n^{-\frac{1}{2}}Y\right|\\
 &  & +\left|\int_{A\cap H_{n}\cap A_{2}}\prod_{i=1}^{n}\frac{\hat{w}_{i}\left(\theta\right)}{\hat{w}_{i}\left(\overline{X}\right)}\rho\left(\theta\right)-\exp\left(-\frac{Y^{T}Y}{2}\right)\sum_{h=0}^{K}\alpha_{h}\left(Y,n\right)n^{-\frac{h}{2}}\diff n^{-\frac{1}{2}}Y\right|.
\end{eqnarray*}


For the first term%
\begin{comment}
change the first second term ref into ref eq
\end{comment}
, by \lemref{exponential-decay-tail}, we have
\begin{eqnarray*}
 &  & \left|\int_{A\cap H_{n}\cap A_{1}}\prod_{i=1}^{n}\frac{\hat{w}_{i}\left(\theta\right)}{\hat{w}_{i}\left(\overline{X}\right)}\rho\left(\theta\right)-\exp\left(-\frac{Y^{T}Y}{2}\right)\sum_{h=0}^{K}\alpha_{h}\left(Y,n\right)n^{-\frac{h}{2}}\diff n^{-\frac{1}{2}}Y\right|\\
 & \le & \int_{A\cap H_{n}\cap A_{1}}\exp\left(n\left(\hat{l}\left(\theta\right)-\hat{l}\left(\overline{X}\right)\right)\right)\rho\left(\theta\right)\diff n^{-\frac{1}{2}}Y\\
 &  & +\left|\int_{A\cap H_{n}\cap A_{1}}\exp\left(-\frac{Y^{T}Y}{4}-\frac{Y^{T}Y}{4}\right)\sum_{h=0}^{K}\alpha_{h}\left(Y,n\right)n^{-\frac{h}{2}}\diff n^{-\frac{1}{2}}Y\right|\\
 & \le & \exp\left(-n\varepsilon\right)\int_{A\cap H_{n}\cap A_{1}}\rho\left(\theta\right)\diff B\left(\theta-\overline{X}\right)\\
 &  & +\exp\left(-\frac{\delta_{2}^{2}n}{4}\right)\left|\int_{A\cap H_{n}\cap A_{1}}\exp\left(-\frac{Y^{T}Y}{4}\right)\sum_{h=0}^{K}\alpha_{h}\left(Y,n\right)n^{-\frac{h}{2}}\diff n^{-\frac{1}{2}}Y\right|\\
 & \le & \exp\left(-n\varepsilon\right)\int_{\mathbb{R}}\rho\left(\theta\right)\diff B\left(\theta-\overline{X}\right)+\exp\left(-n\frac{\delta_{2}^{2}}{4}\right)\sum_{h=0}^{K}\left(\int_{A\cap H_{n}}\exp\left(-\frac{Y^{T}Y}{4}\right)\left|\alpha_{h}\left(Y,n\right)\right|\diff Y\right)n^{-\frac{h+1}{2}}.
\end{eqnarray*}
Note that the above terms are exponentially decreasing with respect
to $n$, so there exists an $N_{3}$, and $M_{3}$, such that for
any $n\ge N_{3}$, 
\[
\left|\int_{A\cap H_{n}\cap A_{1}}\prod_{i=1}^{n}\frac{\hat{w}_{i}\left(\theta\right)}{\hat{w}_{i}\left(\overline{X}\right)}\rho\left(\theta\right)-\exp\left(-\frac{Y^{T}Y}{2}\right)\sum_{h=0}^{K}\alpha_{h}\left(Y,n\right)n^{-\frac{h}{2}}\diff n^{-\frac{1}{2}}Y\right|\le M_{3}n^{-\frac{K+2}{2}}.
\]


For the second term, by \lemref{central-expansion-post-prod}, we
have 
\begin{eqnarray*}
 &  & \left|\int_{A\cap H_{n}\cap A_{2}}\prod_{i=1}^{n}\frac{\hat{w}_{i}\left(\theta\right)}{\hat{w}_{i}\left(\overline{X}\right)}\rho\left(\theta\right)-\exp\left(-\frac{Y^{T}Y}{2}\right)\sum_{h=0}^{K}\alpha_{h}\left(Y,n\right)n^{-\frac{h}{2}}\diff n^{-\frac{1}{2}}Y\right|\\
 & \le & \left|\int_{A\cap H_{n}\cap A_{2}}\prod_{i=1}^{n}\frac{\hat{w}_{i}\left(\theta\right)}{\hat{w}_{i}\left(\overline{X}\right)}\rho\left(\theta\right)-\exp\left(-\frac{1}{2}Y^{T}Y+\sum_{k=3}^{K+3}\delta_{k}\hat{l}n^{-\frac{k-2}{2}}\right)\rho_{K}\left(\theta\right)\diff n^{-\frac{1}{2}}Y\right|\\
 &  & +\left|\int_{A\cap H_{n}\cap A_{2}}\exp\left(-\frac{1}{2}Y^{T}Y+\sum_{k=3}^{K+3}\delta_{k}\hat{l}n^{-\frac{k-2}{2}}\right)\rho_{K}\left(\theta\right)-\exp\left(-\frac{Y^{T}Y}{2}\right)\sum_{h=0}^{K}\alpha_{h}\left(Y,n\right)n^{-\frac{h}{2}}\diff n^{-\frac{1}{2}}Y\right|\\
 & \le & M_{2}n^{-\frac{K+2}{2}}\\
 &  & +\left|\int_{A\cap H_{n}\cap A_{2}}\exp\left(-\frac{1}{2}Y^{T}Y+\sum_{k=3}^{K+3}\delta_{k}\hat{l}n^{-\frac{k-2}{2}}\right)\rho_{K}\left(\theta\right)-\exp\left(-\frac{Y^{T}Y}{2}\right)\sum_{h=0}^{K}\alpha_{h}\left(Y,n\right)n^{-\frac{h}{2}}\diff n^{-\frac{1}{2}}Y\right|.
\end{eqnarray*}
For the second term of above, we add and subtract $R_{K}\left(Y\right)$
in integrand, and by Taylor expansion, 
\begin{eqnarray*}
 &  & \left|\int_{A\cap H_{n}\cap A_{2}}\exp\left(-\frac{1}{2}Y^{T}Y+\sum_{k=3}^{K+3}\delta_{k}\hat{l}n^{-\frac{k-2}{2}}\right)\rho_{K}\left(\theta\right)-\exp\left(-\frac{Y^{T}Y}{2}\right)\sum_{h=0}^{K}\alpha_{h}\left(Y,n\right)n^{-\frac{h}{2}}\diff n^{-\frac{1}{2}}Y\right|\\
 & \le & \left|\int_{A\cap H_{n}\cap A_{2}}\exp\left(-\frac{Y^{T}Y}{2}\right)\rho_{K}\left(\theta\right)\left[\exp\left(\sum_{k=3}^{K+3}\delta_{k}\hat{l}n^{-\frac{k-2}{2}}\right)-\sum_{i=0}^{K+1}\frac{1}{i!}\left(\sum_{k=3}^{K+3}\delta_{k}\hat{l}n^{-\frac{k-2}{2}}\right)^{i}\right]\diff n^{-\frac{1}{2}}Y\right|\\
 &  & +\left|\int_{A\cap H_{n}\cap A_{2}}\exp\left(-\frac{Y^{T}Y}{2}\right)R_{K}\left(Y\right)\diff n^{-\frac{1}{2}}Y\right|\\
 & = & \left|\int_{A\cap H_{n}\cap A_{2}}\exp\left(-\frac{Y^{T}Y}{2}\right)\rho_{K}\left(\theta\right)\frac{1}{\left(K+2\right)!}\exp\left(L\right)\left(\sum_{k=3}^{K+3}\delta_{k}\hat{l}n^{-\frac{k-2}{2}}\right)^{K+2}\diff n^{-\frac{1}{2}}Y\right|\\
 &  & +\left|\int_{A\cap H_{n}\cap A_{2}}\exp\left(-\frac{Y^{T}Y}{2}\right)R_{K}\left(Y\right)\diff n^{-\frac{1}{2}}Y\right|,
\end{eqnarray*}
where $\left|L\right|\le\left|\sum_{k=3}^{K+3}\delta_{k}\hat{l}n^{-\frac{k-2}{2}}\right|$.
We know that $R_{K}\left(Y\right)$ is a polynomial with order $n^{-\frac{1}{2}\left(K+1\right)}$,
so there exists an $M_{3}$, such that 
\[
\left|\int_{A\cap H_{n}\cap A_{2}}\exp\left(-\frac{Y^{T}Y}{2}\right)R_{K}\left(Y\right)\diff n^{-\frac{1}{2}}Y\right|\le M_{3}n^{-\frac{1}{2}\left(K+2\right)}.
\]
For the first term, 

\begin{eqnarray}
 &  & \left|\int_{A\cap H_{n}\cap A_{2}}\exp\left(-\frac{Y^{T}Y}{2}\right)\rho_{K}\left(\theta\right)\frac{1}{\left(K+2\right)!}\exp\left(L\right)\left(\sum_{k=3}^{K+3}\delta_{k}\hat{l}n^{-\frac{k-2}{2}}\right)^{K+2}\diff n^{-\frac{1}{2}}Y\right|\label{eq:1}\\
 & \le & \frac{1}{\left(K+2\right)!}\left|\int_{A\cap H_{n}\cap A_{2}}\exp\left(-\frac{Y^{T}Y}{2}\right)\rho_{K}\left(\theta\right)\exp\left(\left|\sum_{k=3}^{K+3}\delta_{k}\hat{l}n^{-\frac{k-2}{2}}\right|\right)\left(\sum_{k=3}^{K+3}\delta_{k}\hat{l}n^{-\frac{k-2}{2}}\right)^{K+2}\diff n^{-\frac{1}{2}}Y\right|\nonumber \\
 & = & \frac{1}{\left(K+2\right)!}\Bigg|\int_{A\cap H_{n}\cap A_{2}}\rho_{K}\left(\theta\right)\exp\left(-n\left\{ \frac{\left(\theta-\overline{X}\right)^{T}B^{2}\left(\theta-\overline{X}\right)}{2}-\left|\sum_{k=3}^{K+3}\frac{\left[\left(\theta-\overline{X}\right)^{T}\nabla\right]^{k}\hat{l}}{k!}\right|\right\} \right)\nonumber \\
 &  & \left\{ n\sum_{k=3}^{K+3}\frac{\left[\left(\theta-\overline{X}\right)^{T}\nabla\right]^{k}\hat{l}}{k!}\right\} ^{K+2}\diff B\left(\theta-\overline{X}\right)\Bigg|.\nonumber 
\end{eqnarray}
We need $\delta_{2}$sufficiently small, so that there exist an $C_{4}$
and $C_{5}$, such that 
\begin{eqnarray*}
\frac{\left(\theta-\overline{X}\right)^{T}B^{2}\left(\theta-\overline{X}\right)}{2}-\left|\sum_{k=3}^{K+3}\frac{\left[\left(\theta-\overline{X}\right)^{T}\nabla\right]^{k}\hat{l}}{k!}\right| & \ge & C_{4}\left(\theta-\overline{X}\right)^{T}B^{2}\left(\theta-\overline{X}\right),\\
\sum_{k=3}^{K+3}\frac{\left[\left(\theta-\overline{X}\right)^{T}\nabla\right]^{k}\hat{l}}{k!} & \le & C_{5}\left[\left(\theta-\overline{X}\right)^{T}\nabla\right]^{3}\hat{l}.
\end{eqnarray*}
Hence, \eqref{1} can be bounded by 
\begin{eqnarray*}
 &  & \frac{n^{K+2}}{\left(K+2\right)!}\left|\int_{A\cap H_{n}\cap A_{n}}\exp\left(-nC_{4}\left(\theta-\overline{X}\right)^{T}B^{2}\left(\theta-\overline{X}\right)\right)\left\{ C_{5}\left[\left(\theta-\overline{X}\right)^{T}\nabla\right]^{3}\hat{l}\right\} ^{K+2}\diff B\left(\theta-\overline{X}\right)\right|\\
 & \le & \frac{C_{5}^{K+2}n^{K+2}}{\left(K+2\right)!}\left|\int_{A\cap H_{n}}\exp\left(-C_{4}Y^{T}Y\right)\left(\delta_{3}\hat{l}\right)^{K+2}n^{-\frac{3\left(K+2\right)}{2}}\diff n^{-\frac{1}{2}}Y\right|\\
 & \le & \frac{C_{5}^{K+2}}{\left(K+2\right)!}\left|\int_{A\cap H_{n}}\exp\left(-C_{4}Y^{T}Y\right)\left(\delta_{3}\hat{l}\right)^{K+2}\diff Y\right|n^{-\frac{K+3}{2}}.
\end{eqnarray*}
Add all the parts together, we get the inequality in %
\begin{comment}
add ref to main theorem
\end{comment}
.
\end{proof}
